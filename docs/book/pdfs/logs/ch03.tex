% Options for packages loaded elsewhere
\PassOptionsToPackage{unicode}{hyperref}
\PassOptionsToPackage{hyphens}{url}
\documentclass[
]{article}
\usepackage{xcolor}
\usepackage{amsmath,amssymb}
\setcounter{secnumdepth}{-\maxdimen} % remove section numbering
\usepackage{iftex}
\ifPDFTeX
  \usepackage[T1]{fontenc}
  \usepackage[utf8]{inputenc}
  \usepackage{textcomp} % provide euro and other symbols
\else % if luatex or xetex
  \usepackage{unicode-math} % this also loads fontspec
  \defaultfontfeatures{Scale=MatchLowercase}
  \defaultfontfeatures[\rmfamily]{Ligatures=TeX,Scale=1}
\fi
\usepackage{lmodern}
\ifPDFTeX\else
  % xetex/luatex font selection
\fi
% Use upquote if available, for straight quotes in verbatim environments
\IfFileExists{upquote.sty}{\usepackage{upquote}}{}
\IfFileExists{microtype.sty}{% use microtype if available
  \usepackage[]{microtype}
  \UseMicrotypeSet[protrusion]{basicmath} % disable protrusion for tt fonts
}{}
\makeatletter
\@ifundefined{KOMAClassName}{% if non-KOMA class
  \IfFileExists{parskip.sty}{%
    \usepackage{parskip}
  }{% else
    \setlength{\parindent}{0pt}
    \setlength{\parskip}{6pt plus 2pt minus 1pt}}
}{% if KOMA class
  \KOMAoptions{parskip=half}}
\makeatother
\usepackage{color}
\usepackage{fancyvrb}
\newcommand{\VerbBar}{|}
\newcommand{\VERB}{\Verb[commandchars=\\\{\}]}
\DefineVerbatimEnvironment{Highlighting}{Verbatim}{commandchars=\\\{\}}
% Add ',fontsize=\small' for more characters per line
\newenvironment{Shaded}{}{}
\newcommand{\AlertTok}[1]{\textcolor[rgb]{1.00,0.00,0.00}{\textbf{#1}}}
\newcommand{\AnnotationTok}[1]{\textcolor[rgb]{0.38,0.63,0.69}{\textbf{\textit{#1}}}}
\newcommand{\AttributeTok}[1]{\textcolor[rgb]{0.49,0.56,0.16}{#1}}
\newcommand{\BaseNTok}[1]{\textcolor[rgb]{0.25,0.63,0.44}{#1}}
\newcommand{\BuiltInTok}[1]{\textcolor[rgb]{0.00,0.50,0.00}{#1}}
\newcommand{\CharTok}[1]{\textcolor[rgb]{0.25,0.44,0.63}{#1}}
\newcommand{\CommentTok}[1]{\textcolor[rgb]{0.38,0.63,0.69}{\textit{#1}}}
\newcommand{\CommentVarTok}[1]{\textcolor[rgb]{0.38,0.63,0.69}{\textbf{\textit{#1}}}}
\newcommand{\ConstantTok}[1]{\textcolor[rgb]{0.53,0.00,0.00}{#1}}
\newcommand{\ControlFlowTok}[1]{\textcolor[rgb]{0.00,0.44,0.13}{\textbf{#1}}}
\newcommand{\DataTypeTok}[1]{\textcolor[rgb]{0.56,0.13,0.00}{#1}}
\newcommand{\DecValTok}[1]{\textcolor[rgb]{0.25,0.63,0.44}{#1}}
\newcommand{\DocumentationTok}[1]{\textcolor[rgb]{0.73,0.13,0.13}{\textit{#1}}}
\newcommand{\ErrorTok}[1]{\textcolor[rgb]{1.00,0.00,0.00}{\textbf{#1}}}
\newcommand{\ExtensionTok}[1]{#1}
\newcommand{\FloatTok}[1]{\textcolor[rgb]{0.25,0.63,0.44}{#1}}
\newcommand{\FunctionTok}[1]{\textcolor[rgb]{0.02,0.16,0.49}{#1}}
\newcommand{\ImportTok}[1]{\textcolor[rgb]{0.00,0.50,0.00}{\textbf{#1}}}
\newcommand{\InformationTok}[1]{\textcolor[rgb]{0.38,0.63,0.69}{\textbf{\textit{#1}}}}
\newcommand{\KeywordTok}[1]{\textcolor[rgb]{0.00,0.44,0.13}{\textbf{#1}}}
\newcommand{\NormalTok}[1]{#1}
\newcommand{\OperatorTok}[1]{\textcolor[rgb]{0.40,0.40,0.40}{#1}}
\newcommand{\OtherTok}[1]{\textcolor[rgb]{0.00,0.44,0.13}{#1}}
\newcommand{\PreprocessorTok}[1]{\textcolor[rgb]{0.74,0.48,0.00}{#1}}
\newcommand{\RegionMarkerTok}[1]{#1}
\newcommand{\SpecialCharTok}[1]{\textcolor[rgb]{0.25,0.44,0.63}{#1}}
\newcommand{\SpecialStringTok}[1]{\textcolor[rgb]{0.73,0.40,0.53}{#1}}
\newcommand{\StringTok}[1]{\textcolor[rgb]{0.25,0.44,0.63}{#1}}
\newcommand{\VariableTok}[1]{\textcolor[rgb]{0.10,0.09,0.49}{#1}}
\newcommand{\VerbatimStringTok}[1]{\textcolor[rgb]{0.25,0.44,0.63}{#1}}
\newcommand{\WarningTok}[1]{\textcolor[rgb]{0.38,0.63,0.69}{\textbf{\textit{#1}}}}
% definitions for citeproc citations
\NewDocumentCommand\citeproctext{}{}
\NewDocumentCommand\citeproc{mm}{%
  \begingroup\def\citeproctext{#2}\cite{#1}\endgroup}
\makeatletter
 % allow citations to break across lines
 \let\@cite@ofmt\@firstofone
 % avoid brackets around text for \cite:
 \def\@biblabel#1{}
 \def\@cite#1#2{{#1\if@tempswa , #2\fi}}
\makeatother
\newlength{\cslhangindent}
\setlength{\cslhangindent}{1.5em}
\newlength{\csllabelwidth}
\setlength{\csllabelwidth}{3em}
\newenvironment{CSLReferences}[2] % #1 hanging-indent, #2 entry-spacing
 {\begin{list}{}{%
  \setlength{\itemindent}{0pt}
  \setlength{\leftmargin}{0pt}
  \setlength{\parsep}{0pt}
  % turn on hanging indent if param 1 is 1
  \ifodd #1
   \setlength{\leftmargin}{\cslhangindent}
   \setlength{\itemindent}{-1\cslhangindent}
  \fi
  % set entry spacing
  \setlength{\itemsep}{#2\baselineskip}}}
 {\end{list}}
\usepackage{calc}
\newcommand{\CSLBlock}[1]{\hfill\break\parbox[t]{\linewidth}{\strut\ignorespaces#1\strut}}
\newcommand{\CSLLeftMargin}[1]{\parbox[t]{\csllabelwidth}{\strut#1\strut}}
\newcommand{\CSLRightInline}[1]{\parbox[t]{\linewidth - \csllabelwidth}{\strut#1\strut}}
\newcommand{\CSLIndent}[1]{\hspace{\cslhangindent}#1}
\setlength{\emergencystretch}{3em} % prevent overfull lines
\providecommand{\tightlist}{%
  \setlength{\itemsep}{0pt}\setlength{\parskip}{0pt}}
% LaTeX preamble for PDF compilation
% Minimal configuration - source files use proper LaTeX commands

\usepackage[most]{tcolorbox}

% =============================================================================
% CALLOUT / ADMONITION BOXES (tcolorbox)
%
% Used by:
% - docs/book/callouts.lua  (::: fenced div callouts)
% - docs/book/admonitions.lua (MkDocs !!! / ??? admonitions)
% =============================================================================

% Note box (blue) - matches callouts.lua
\newtcolorbox{CalloutNote}[1]{
  breakable,
  colback=blue!4!white,
  colframe=blue!40!black,
  boxrule=0.5pt,
  arc=2pt,
  title=#1,
  fonttitle=\bfseries\sffamily
}

% MkDocs admonitions use NoteBox (keep it as an alias).
\newtcolorbox{NoteBox}[1]{
  breakable,
  colback=blue!4!white,
  colframe=blue!40!black,
  boxrule=0.5pt,
  arc=2pt,
  title=#1,
  fonttitle=\bfseries\sffamily
}

% Tip box (green)
\newtcolorbox{TipBox}[1]{
  breakable,
  colback=green!4!white,
  colframe=green!40!black,
  boxrule=0.5pt,
  arc=2pt,
  title=#1,
  fonttitle=\bfseries\sffamily
}

% Warning box (orange)
\newtcolorbox{WarningBox}[1]{
  breakable,
  colback=orange!4!white,
  colframe=orange!60!black,
  boxrule=0.5pt,
  arc=2pt,
  title=#1,
  fonttitle=\bfseries\sffamily
}

% Danger box (red)
\newtcolorbox{DangerBox}[1]{
  breakable,
  colback=red!4!white,
  colframe=red!50!black,
  boxrule=0.5pt,
  arc=2pt,
  title=#1,
  fonttitle=\bfseries\sffamily
}

% Info box (cyan)
\newtcolorbox{InfoBox}[1]{
  breakable,
  colback=cyan!4!white,
  colframe=cyan!40!black,
  boxrule=0.5pt,
  arc=2pt,
  title=#1,
  fonttitle=\bfseries\sffamily
}

% Example box (purple)
\newtcolorbox{ExampleBox}[1]{
  breakable,
  colback=violet!4!white,
  colframe=violet!40!black,
  boxrule=0.5pt,
  arc=2pt,
  title=#1,
  fonttitle=\bfseries\sffamily
}

% Important box (magenta)
\newtcolorbox{ImportantBox}[1]{
  breakable,
  colback=magenta!4!white,
  colframe=magenta!40!black,
  boxrule=0.5pt,
  arc=2pt,
  title=#1,
  fonttitle=\bfseries\sffamily
}

% Caution box (yellow)
\newtcolorbox{CautionBox}[1]{
  breakable,
  colback=yellow!10!white,
  colframe=yellow!60!black,
  boxrule=0.5pt,
  arc=2pt,
  title=#1,
  fonttitle=\bfseries\sffamily
}

% Abstract box (gray)
\newtcolorbox{AbstractBox}[1]{
  breakable,
  colback=gray!4!white,
  colframe=gray!40!black,
  boxrule=0.5pt,
  arc=2pt,
  title=#1,
  fonttitle=\bfseries\sffamily
}

% Question box (teal)
\newtcolorbox{QuestionBox}[1]{
  breakable,
  colback=teal!4!white,
  colframe=teal!40!black,
  boxrule=0.5pt,
  arc=2pt,
  title=#1,
  fonttitle=\bfseries\sffamily
}

% Quote box (light gray)
\newtcolorbox{QuoteBox}[1]{
  breakable,
  colback=gray!2!white,
  colframe=gray!30!black,
  boxrule=0.5pt,
  arc=0pt,
  leftrule=3pt,
  title=#1,
  fonttitle=\bfseries\sffamily\itshape
}
\usepackage{bookmark}
\IfFileExists{xurl.sty}{\usepackage{xurl}}{} % add URL line breaks if available
\urlstyle{same}
\hypersetup{
  hidelinks,
  pdfcreator={LaTeX via pandoc}}

\author{}
\date{}

\begin{document}

\section{Chapter 3 --- Stochastic Processes and Bellman
Foundations}\label{chapter-3-stochastic-processes-and-bellman-foundations}

\emph{Vlad Prytula}

\subsection{3.1 Motivation: From Single Queries to Sequential
Sessions}\label{motivation-from-single-queries-to-sequential-sessions}

Chapter 1 formalized search ranking as a contextual bandit: observe
context \(x\) (user segment, query type), select action \(a\) (boost
weights), and observe an immediate reward \(R(x, a, \omega)\). This
abstraction is appropriate when each query can be treated as an
independent decision, and when myopic objectives (GMV, CM2, clicks) are
sufficient proxies for long-run value.

In deployed systems, user behavior is sequential. Actions taken on one
query influence the distribution of future queries, clicks, and
purchases within a session, and they can shape return probability across
sessions. A user may refine a query after inspecting a ranking; a cart
may accumulate over several steps; satisfaction may drift and eventually
trigger abandonment. These are precisely the phenomena that a
single-step model cannot represent.

Mathematically, the missing ingredient is \textbf{state}: a variable
\(S_t\) that summarizes the relevant history at time \(t\) (cart,
browsing context, latent satisfaction, recency). Once we represent the
interaction as a controlled stochastic process
\((S_0, A_0, R_0, S_1, A_1, R_1, \ldots)\), the central objects of
reinforcement learning become well-defined:

\begin{itemize}
\tightlist
\item
  \textbf{Value functions} \(V^\pi(s)\) and \(Q^\pi(s,a)\), which
  measure expected cumulative reward under a policy \(\pi\)
\item
  \textbf{Bellman operators} \(\mathcal{T}^\pi\) and \(\mathcal{T}\),
  which encode the dynamic programming principle as fixed-point
  equations
\end{itemize}

The guiding question of this chapter is structural: under what
assumptions does repeated Bellman backup converge, and why does it
converge to the optimal value function? The answer is an
operator-theoretic one: the discounted Bellman operator is a contraction
in the sup-norm, so it has a unique fixed point and value iteration
converges to it.

We develop these foundations in the following order:

\begin{enumerate}
\def\labelenumi{\arabic{enumi}.}
\tightlist
\item
  Section 3.2--3.3: Stochastic processes, filtrations, stopping times
  (measure-theoretic rigor for sequential randomness)
\item
  Section 3.4: Markov Decision Processes (formal definition, standard
  Borel assumptions)
\item
  Section 3.5: Bellman operators and value functions (from intuition to
  operators on function spaces)
\item
  Section 3.6: Contraction mappings and Banach fixed-point theorem
  (complete proof, step-by-step)
\item
  Section 3.7: Value iteration convergence (why dynamic programming
  works)
\item
  Section 3.8: Connection to bandits (the \(\gamma = 0\) special case
  from Chapter 1)
\item
  Section 3.9: Computational verification (NumPy experiments)
\item
  Section 3.10: RL bridges (preview of Chapter 11's multi-episode
  formulation)
\end{enumerate}

By the end of this chapter, we understand:

\begin{itemize}
\tightlist
\item
  Why value iteration converges exponentially fast (contraction mapping
  theorem)
\item
  How to prove convergence of RL algorithms rigorously (fixed-point
  theory)
\item
  When the bandit formulation is sufficient and when MDPs are necessary
\item
  The mathematical foundations that justify TD-learning, Q-learning, and
  policy gradients
\end{itemize}

Prerequisites. This chapter assumes:

\begin{itemize}
\tightlist
\item
  Measure-theoretic probability from Chapter 2 (probability spaces,
  random variables, conditional expectation)
\item
  Familiarity with supremum norm and function spaces (we will introduce
  contraction mappings from first principles)
\end{itemize}

\begin{center}\rule{0.5\linewidth}{0.5pt}\end{center}

\subsection{3.2 Stochastic Processes: Modeling Sequential
Randomness}\label{stochastic-processes-modeling-sequential-randomness}

\textbf{Definition 3.2.1} (Stochastic Process)
\phantomsection\label{DEF-3.2.1}

Let \((\Omega, \mathcal{F}, \mathbb{P})\) be a probability space,
\(T \subseteq \mathbb{R}_+\) an index set (often \(T = \mathbb{N}\) or
\(T = [0, \infty)\)), and \((E, \mathcal{E})\) a measurable space. A
\textbf{stochastic process} is a collection of random variables
\(\{X_t : t \in T\}\) where each \(X_t: \Omega \to E\) is
\((\mathcal{F}, \mathcal{E})\)-measurable.

\textbf{Notation}: We write \((X_t)_{t \in T}\) or simply \((X_t)\) when
\(T\) is clear from context.

\textbf{Intuition}: A stochastic process is a \textbf{time-indexed
family of random variables}. Each \(X_t\) represents the state of a
system at time \(t\). For a fixed \(\omega \in \Omega\), the mapping
\(t \mapsto X_t(\omega)\) is a \textbf{sample path} or
\textbf{trajectory}.

\textbf{Example 3.2.1} (User satisfaction process). Let \(E = [0, 1]\)
represent satisfaction levels. Define \(S_t: \Omega \to [0, 1]\) as the
user's satisfaction after the \(t\)-th query in a session. Then
\((S_t)_{t=0}^T\) is a stochastic process modeling satisfaction
evolution.

\textbf{Example 3.2.2} (RL trajectory). In a Markov Decision Process,
the sequence \((S_0, A_0, R_0, S_1, A_1, R_1, \ldots)\) is a stochastic
process where: - \(S_t \in \mathcal{S}\) (state space) -
\(A_t \in \mathcal{A}\) (action space) - \(R_t \in \mathbb{R}\) (reward)

Each component is a random variable, and their joint distribution is
induced by the policy \(\pi\) and environment dynamics \(P\).

\textbf{Standing convention (Discrete time).} Throughout this chapter we
work in discrete time with index set \(T=\mathbb{N}\). Continuous-time
analogues require additional measurability notions (e.g., predictable
processes and optional \(\sigma\)-algebras), which we do not pursue
here.

\begin{center}\rule{0.5\linewidth}{0.5pt}\end{center}

\subsubsection{3.2.1 Filtrations and Adapted
Processes}\label{filtrations-and-adapted-processes}

\textbf{Definition 3.2.2} (Filtration) \phantomsection\label{DEF-3.2.2}

A \textbf{filtration} on \((\Omega, \mathcal{F}, \mathbb{P})\) is a
collection \((\mathcal{F}_t)_{t \in T}\) of sub-\(\sigma\)-algebras of
\(\mathcal{F}\) satisfying: \[
\mathcal{F}_s \subseteq \mathcal{F}_t \subseteq \mathcal{F} \quad \text{for all } s \leq t.
\]

\textbf{Intuition}: \(\mathcal{F}_t\) represents the \textbf{information
available at time \(t\)}. The inclusion
\(\mathcal{F}_s \subseteq \mathcal{F}_t\) captures the idea that
information accumulates over time: we never ``forget'' past
observations.

\textbf{Example 3.2.3} (Natural filtration). Given a stochastic process
\((X_t)\), the \textbf{natural filtration} is: \[
\mathcal{F}_t := \sigma(X_s : s \leq t),
\] the smallest \(\sigma\)-algebra making all \(X_s\) with \(s \leq t\)
measurable. This represents ``all information revealed by observing
\((X_0, X_1, \ldots, X_t)\).''

We write \[
\mathcal{F}_\infty := \sigma\left(\bigcup_{t \in T} \mathcal{F}_t\right)
\] for the terminal \(\sigma\)-algebra generated by the filtration.

\textbf{Definition 3.2.3} (Adapted Process)
\phantomsection\label{DEF-3.2.3}

A stochastic process \((X_t)\) is \textbf{adapted} to the filtration
\((\mathcal{F}_t)\) if \(X_t\) is \(\mathcal{F}_t\)-measurable for all
\(t \in T\).

\textbf{Intuition}: Adaptedness means ``the value of \(X_t\) is
determined by information available at time \(t\).'' This is the
mathematical formalization of \textbf{causality}: \(X_t\) cannot depend
on future information \(\mathcal{F}_s\) with \(s > t\).

\textbf{Remark 3.2.1} (Adapted vs.~predictable). In continuous-time
stochastic calculus, there is a stronger notion called
\textbf{predictable} (measurable with respect to \(\mathcal{F}_{t-}\),
the left limit). For discrete-time RL, adapted suffices.

\textbf{Remark 3.2.2} (RL policies must be adapted). In reinforcement
learning, a policy \(\pi(a | h_t)\) at time \(t\) must depend only on
the \textbf{history} \(h_t = (s_0, a_0, r_0, \ldots, s_t)\) available at
\(t\), not on future states \(s_{t+1}, s_{t+2}, \ldots\). This is
precisely the adaptedness condition: \(\pi_t\) is
\(\mathcal{F}_t\)-measurable where \(\mathcal{F}_t = \sigma(h_t)\).

\begin{center}\rule{0.5\linewidth}{0.5pt}\end{center}

\subsubsection{3.2.2 Stopping Times}\label{stopping-times}

\textbf{Definition 3.2.4} (Stopping Time)
\phantomsection\label{DEF-3.2.4}

Let \((\mathcal{F}_t)\) be a filtration on
\((\Omega, \mathcal{F}, \mathbb{P})\). A random variable
\(\tau: \Omega \to T \cup \{\infty\}\) is a \textbf{stopping time} if:
\[
\{\tau \leq t\} \in \mathcal{F}_t \quad \text{for all } t \in T.
\]

\textbf{Intuition}: A stopping time is a \textbf{random time} whose
occurrence is determined by information available \emph{up to that
time}. The event ``\(\tau\) has occurred by time \(t\)'' must be
\(\mathcal{F}_t\)-measurable---we can decide whether to stop using only
observations \((X_0, \ldots, X_t)\), without peeking into the future.

\textbf{Example 3.2.4} (Session abandonment). Define: \[
\tau := \inf\{t \geq 0 : S_t < \theta\},
\] the first time user satisfaction \(S_t\) drops below threshold
\(\theta\). This is a stopping time: to check ``\(\tau \leq t\)'' (user
has abandoned by time \(t\)), we only need to observe
\((S_0, \ldots, S_t)\). We do not need to know future satisfaction
\(S_{t+1}, S_{t+2}, \ldots\).

\textbf{Example 3.2.5} (Purchase event). Define \(\tau\) as the first
time the user makes a purchase. This is a stopping time: the event
``\(\tau = t\)'' means ``user purchased at time \(t\), having not
purchased before''---determined by history up to \(t\).

\textbf{Non-Example 3.2.6} (Last time satisfaction peaks). Define
\(\tau := \sup\{t : S_t = \max_{s \leq T} S_s\}\) (the last time
satisfaction reaches its maximum over \([0, T]\)). This is \textbf{NOT}
a stopping time: to determine ``\(\tau = t\),'' we need to know future
values \(S_{t+1}, \ldots, S_T\) to verify satisfaction never exceeds
\(S_t\) afterward.

\textbf{Proposition 3.2.1} (Measurability of stopped processes)
\phantomsection\label{PROP-3.2.1}

Assume \(T=\mathbb{N}\). If \((X_t)\) is adapted to \((\mathcal{F}_t)\)
and \(\tau\) is a stopping time, then
\(X_\tau \mathbf{1}_{\{\tau < \infty\}}\) is
\(\mathcal{F}_\infty\)-measurable.

\emph{Proof.} \textbf{Step 1} (Indicator decomposition). Write: \[
X_\tau \mathbf{1}_{\{\tau < \infty\}} = \sum_{t=0}^\infty X_t \mathbf{1}_{\{\tau = t\}}.
\]

\textbf{Step 2} (Measurability of indicators). For each
\(t \in \mathbb{N}\), the event \(\{\tau = t\}\) belongs to
\(\mathcal{F}_t\). Indeed,
\(\{\tau = 0\} = \{\tau \leq 0\} \in \mathcal{F}_0\), and for
\(t \geq 1\), \[
\{\tau = t\} = \{\tau \leq t\} \cap \{\tau \leq t-1\}^c \in \mathcal{F}_t,
\] since \(\{\tau \leq t\} \in \mathcal{F}_t\) and
\(\{\tau \leq t-1\} \in \mathcal{F}_{t-1} \subseteq \mathcal{F}_t\).

\textbf{Step 3} (Measurability of \(X_t \mathbf{1}_{\{\tau = t\}}\)).
Since \(X_t\) is \(\mathcal{F}_t\)-measurable and
\(\{\tau = t\} \in \mathcal{F}_t\), the product
\(X_t \mathbf{1}_{\{\tau = t\}}\) is \(\mathcal{F}_t\)-measurable, hence
\(\mathcal{F}_\infty\)-measurable.

\textbf{Step 4} (Countable sum). A countable sum of measurable functions
is measurable, so the right-hand side of Step 1 is
\(\mathcal{F}_\infty\)-measurable; hence
\(X_\tau \mathbf{1}_{\{\tau < \infty\}}\) is
\(\mathcal{F}_\infty\)-measurable. \(\square\)

\textbf{Remark 3.2.3} (The indicator technique). The proof uses the
\textbf{indicator decomposition}: write a stopped process as a sum over
stopping events. This technique will reappear when proving optional
stopping theorems for martingales (used in stochastic approximation
convergence proofs, deferred to later chapters).

\textbf{Remark 3.2.4} (RL preview: Episode termination). In episodic RL,
the terminal time \(T\) is often a stopping time: the episode ends when
the agent reaches a terminal state (e.g., user completes purchase or
abandons session). The return \(G_0 = \sum_{t=0}^{T-1} \gamma^t R_t\)
depends on \(T\), which is random. Proposition 3.2.1 ensures \(G_0\) is
well-defined as a random variable.

\begin{center}\rule{0.5\linewidth}{0.5pt}\end{center}

\subsection{3.3 Markov Chains and the Markov
Property}\label{markov-chains-and-the-markov-property}

Before defining MDPs, we introduce \textbf{Markov chains}---stochastic
processes with memoryless transitions.

\textbf{Definition 3.3.1} (Markov Chain)
\phantomsection\label{DEF-3.3.1}

A discrete-time stochastic process \((X_t)_{t \in \mathbb{N}}\) taking
values in a countable or general measurable space \((E, \mathcal{E})\)
is a \textbf{Markov chain} (with respect to its natural filtration
\(\mathcal{F}_t := \sigma(X_0, \ldots, X_t)\)) if: \[
\mathbb{P}(X_{t+1} \in A \mid \mathcal{F}_t) = \mathbb{P}(X_{t+1} \in A \mid X_t) \quad \text{for all } A \in \mathcal{E}, \, t \geq 0.
\tag{3.1}
\label{EQ-3.1}\]

This is the \textbf{Markov property}: the future \(X_{t+1}\) is
conditionally independent of the past \((X_0, \ldots, X_{t-1})\) given
the present \(X_t\).

\textbf{Intuition}: ``The future depends on the present, not on how we
arrived at the present.''

\textbf{Example 3.3.1} (Random walk). Let \((\xi_t)\) be i.i.d. random
variables with
\(\mathbb{P}(\xi_t = +1) = \mathbb{P}(\xi_t = -1) = 1/2\). Define
\(X_t = \sum_{s=0}^{t-1} \xi_s\) (cumulative sum). Then: \[
X_{t+1} = X_t + \xi_t,
\] so \(X_{t+1}\) depends only on \(X_t\) and the new increment
\(\xi_t\) (independent of history). This is a Markov chain.

\textbf{Example 3.3.2} (User state transitions). In an e-commerce
session, let
\(X_t \in \{\text{browsing}, \text{engaged}, \text{ready\_to\_buy}, \text{abandoned}\}\)
be the user's state after \(t\) queries. If transitions depend only on
current state (not on the path taken to reach it), then \((X_t)\) is a
Markov chain.

\textbf{Non-Example 3.3.3} (ARMA processes violate the Markov property).
Consider \(X_t = 0.5 X_{t-1} + 0.3 X_{t-2} + \varepsilon_t\) where
\(\varepsilon_t\) is white noise. Given only \(X_t\), the distribution
of \(X_{t+1}\) depends on \(X_{t-1}\) (through the \(0.3\) term),
violating the Markov property \eqref{EQ-3.1}. To restore Markovianity,
\textbf{augment the state}: define \(\tilde{X}_t = (X_t, X_{t-1})\).
Then \(\tilde{X}_{t+1}\) depends only on \(\tilde{X}_t\), so
\((\tilde{X}_t)\) is Markov. This \textbf{state augmentation} technique
is fundamental in RL---frame stacking in video games (Remark 3.4.1),
LSTM hidden states, and user history embeddings all restore the Markov
property by expanding what we call ``state.''

\textbf{Definition 3.3.2} (Transition Kernel)
\phantomsection\label{DEF-3.3.2}

The \textbf{transition kernel} (or \textbf{transition probability}) of a
Markov chain is: \[
P(x, A) := \mathbb{P}(X_{t+1} \in A \mid X_t = x), \quad x \in E, \, A \in \mathcal{E}.
\]

For time-homogeneous chains, \(P\) is independent of \(t\).

\textbf{Properties}: 1. For each \(x \in E\), \(A \mapsto P(x, A)\) is a
probability measure on \((E, \mathcal{E})\) 2. For each
\(A \in \mathcal{E}\), \(x \mapsto P(x, A)\) is measurable

\textbf{Remark 3.3.1} (Standard Borel assumption). For general state
spaces, we require \(E\) to be a \textbf{standard Borel space} (a
measurable subset of a Polish space, i.e., separable complete metric
space). This ensures:

\begin{itemize}
\tightlist
\item
  Transition kernels \(P(x, A)\) are well-defined and measurable
\item
  Regular conditional probabilities exist
\item
  Optimal policies can be chosen measurably
\end{itemize}

All finite and countable spaces are standard Borel. \(\mathbb{R}^n\)
with Borel \(\sigma\)-algebra is standard Borel. This covers essentially
all RL applications.

\begin{center}\rule{0.5\linewidth}{0.5pt}\end{center}

\subsection{3.4 Markov Decision Processes: The RL
Framework}\label{markov-decision-processes-the-rl-framework}

\textbf{Definition 3.4.1} (Markov Decision Process)
\phantomsection\label{DEF-3.4.1}

A \textbf{Markov Decision Process (MDP)} is a tuple
\((\mathcal{S}, \mathcal{A}, P, R, \gamma)\) where:

\begin{enumerate}
\def\labelenumi{\arabic{enumi}.}
\tightlist
\item
  \(\mathcal{S}\) is the \textbf{state space} (a standard Borel space)
\item
  \(\mathcal{A}\) is the \textbf{action space} (a standard Borel space)
\item
  \(P(\cdot \mid s,a)\) is a \textbf{Markov kernel} from
  \((\mathcal{S}\times\mathcal{A}, \mathcal{B}(\mathcal{S})\otimes\mathcal{B}(\mathcal{A}))\)
  to \((\mathcal{S}, \mathcal{B}(\mathcal{S}))\): \(P(B \mid s,a)\) is
  the probability of transitioning to a Borel set
  \(B \subseteq \mathcal{S}\) when taking action \(a\) in state \(s\)
\item
  \(R: \mathcal{S} \times \mathcal{A} \times \mathcal{S} \to \mathbb{R}\)
  is the \textbf{reward function}: \(R(s, a, s')\) is the reward
  obtained from transition \((s, a, s')\)
\item
  \(\gamma \in [0, 1)\) is the \textbf{discount factor}
\end{enumerate}

\textbf{Structural assumptions}: - For each \((s, a)\),
\(B \mapsto P(B \mid s, a)\) is a probability measure on
\((\mathcal{S}, \mathcal{B}(\mathcal{S}))\) - For each
\(B \in \mathcal{B}(\mathcal{S})\), \((s, a) \mapsto P(B \mid s, a)\) is
measurable - \(R\) is bounded and measurable:
\(|R(s, a, s')| \leq R_{\max} < \infty\) for all \((s, a, s')\)

\textbf{Notation}: - We write the bounded measurable one-step expected
reward as \[
  r(s,a) := \int_{\mathcal{S}} R(s,a,s')\,P(ds'\mid s,a).
  \] When \(\mathcal{S}\) is finite, integrals against
\(P(\cdot\mid s,a)\) reduce to sums:
\(\int_{\mathcal{S}} f(s')\,P(ds'\mid s,a)=\sum_{s'\in\mathcal{S}}P(s'\mid s,a)f(s')\).
- When \(\mathcal{S}\) and \(\mathcal{A}\) are finite, we represent
\(P\) as a tensor
\(P \in [0,1]^{|\mathcal{S}| \times |\mathcal{A}| \times |\mathcal{S}|}\)
with \(P_{s,a,s'} = P(s' \mid s, a)\)

\textbf{Definition 3.4.2} (Policy) \phantomsection\label{DEF-3.4.2}

A \textbf{(stationary Markov) policy} is a \textbf{stochastic kernel}
\(\pi(\cdot \mid s)\) from \((\mathcal{S}, \mathcal{B}(\mathcal{S}))\)
to \((\mathcal{A}, \mathcal{B}(\mathcal{A}))\) such that: - For each
\(s \in \mathcal{S}\), \(B \mapsto \pi(B \mid s)\) is a probability
measure on \((\mathcal{A}, \mathcal{B}(\mathcal{A}))\) - For each
\(B \in \mathcal{B}(\mathcal{A})\), \(s \mapsto \pi(B \mid s)\) is
\(\mathcal{B}(\mathcal{S})\)-measurable

We write integrals against the policy as \(\pi(da \mid s)\). When
\(\mathcal{A}\) is finite, \(\pi(a \mid s)\) is a mass function and
\(\int_{\mathcal{A}} f(a)\,\pi(da\mid s) = \sum_{a\in\mathcal{A}} f(a)\,\pi(a\mid s)\).

\textbf{Deterministic policies}: A policy is deterministic if
\(\pi(\cdot \mid s)\) is a point mass for all \(s\). We identify
deterministic policies with measurable functions
\(\pi: \mathcal{S} \to \mathcal{A}\), with the induced kernel
\(\pi(da\mid s) = \delta_{\pi(s)}(da)\).

\textbf{Assumption 3.4.1} (Markov assumption).
\phantomsection\label{ASM-3.4.1}

The MDP satisfies: 1. \textbf{Transition Markov property}:
\(\mathbb{P}(S_{t+1} \in B \mid s_0, a_0, \ldots, s_t, a_t) = P(B \mid s_t, a_t)\)
2. \textbf{Reward Markov property}:
\(\mathbb{E}[R_t \mid s_0, a_0, \ldots, s_t, a_t, s_{t+1}] = R(s_t, a_t, s_{t+1})\)

These properties ensure that \textbf{state \(s_t\) summarizes all past
information relevant for predicting the future}. This is crucial: if the
state does not satisfy the Markov property, the MDP framework breaks
down.

\textbf{Remark 3.4.1} (State design in practice). Real systems rarely
have perfectly Markovian observations. Practitioners construct
\textbf{augmented states} to restore the Markov property:

\begin{itemize}
\tightlist
\item
  \textbf{Frame stacking} in video games: stack last 4 frames to capture
  velocity
\item
  \textbf{LSTM hidden states}: recurrent network state becomes part of
  MDP state
\item
  \textbf{User history embeddings}: include session features (past
  clicks, queries) in context vector
\end{itemize}

This is a modeling choice rather than a theorem, but it is essential for
applying MDP theory in practice.

\begin{center}\rule{0.5\linewidth}{0.5pt}\end{center}

\subsubsection{3.4.1 Value Functions}\label{value-functions}

\textbf{Definition 3.4.3} (State-Value Function)
\phantomsection\label{DEF-3.4.3}

Given a policy \(\pi\) and initial state \(s \in \mathcal{S}\), the
\textbf{state-value function} is: \[
V^\pi(s) := \mathbb{E}^\pi\left[\sum_{t=0}^\infty \gamma^t R_t \,\bigg|\, S_0 = s\right],
\tag{3.2}
\label{EQ-3.2}\]

where the expectation is over trajectories
\((S_0, A_0, R_0, S_1, A_1, R_1, \ldots)\) generated by policy \(\pi\)
and transition kernel \(P\): - \(S_0 = s\) (initial state) -
\(A_t \sim \pi(\cdot | S_t)\) (actions sampled from policy) -
\(S_{t+1} \sim P(\cdot | S_t, A_t)\) (states transition according to
dynamics) - \(R_t = R(S_t, A_t, S_{t+1})\) (rewards realized from
transitions)

\textbf{Notation}: The superscript \(\mathbb{E}^\pi\) emphasizes that
the expectation is under the probability measure \(\mathbb{P}^\pi\)
induced by policy \(\pi\) and dynamics \(P\). Existence and uniqueness
of this trajectory measure (built from the initial state, the policy
kernel, and the transition kernel) can be formalized via the
Ionescu--Tulcea extension theorem; Chapter 2 gives the measurable
construction of MDP trajectories.

\textbf{Well-definedness}: Since \(\gamma < 1\) and
\(|R_t| \leq R_{\max}\), the series converges absolutely: \[
\left|\sum_{t=0}^\infty \gamma^t R_t\right| \leq \sum_{t=0}^\infty \gamma^t R_{\max} = \frac{R_{\max}}{1 - \gamma} < \infty.
\]

Thus \(V^\pi(s)\) is well-defined and
\(|V^\pi(s)| \leq R_{\max}/(1-\gamma)\) for all \(s, \pi\).

\textbf{Definition 3.4.4} (Action-Value Function)
\phantomsection\label{DEF-3.4.4}

Given a policy \(\pi\), state \(s\), and action \(a\), the
\textbf{action-value function} (or \textbf{Q-function}) is: \[
Q^\pi(s, a) := \mathbb{E}^\pi\left[\sum_{t=0}^\infty \gamma^t R_t \,\bigg|\, S_0 = s, A_0 = a\right].
\tag{3.3}
\label{EQ-3.3}\]

This is the expected return starting from state \(s\), taking action
\(a\), then following policy \(\pi\).

\textbf{Relationship}: \[
V^\pi(s) = \int_{\mathcal{A}} Q^\pi(s, a)\,\pi(da \mid s),
\tag{3.4}
\label{EQ-3.4}\]

When \(\mathcal{A}\) is finite, the integral reduces to the familiar sum
\(\sum_{a\in\mathcal{A}} \pi(a\mid s) Q^\pi(s,a)\).

\textbf{Definition 3.4.5} (Optimal Value Functions)
\phantomsection\label{DEF-3.4.5}

The \textbf{optimal state-value function} is: \[
V^*(s) := \sup_\pi V^\pi(s),
\tag{3.5}
\label{EQ-3.5}\]

and the \textbf{optimal action-value function} is: \[
Q^*(s, a) := \sup_\pi Q^\pi(s, a).
\tag{3.6}
\label{EQ-3.6}\]

\textbf{Remark 3.4.2} (Existence of optimal policies and measurable
selection). In finite action spaces, optimal actions exist statewise and
the Bellman optimality operator can be written with \(\max\). In general
Borel state-action spaces, the optimality operator is stated with
\(\sup\), and existence of a \textbf{deterministic stationary} optimal
policy can require additional topological conditions (e.g., compact
\(\mathcal{A}\) and upper semicontinuity) or a measurable selection
theorem.

Chapter 2 discusses this fine print and gives a measurable formulation
of the Bellman operators; see also (Puterman 2014, Theorem 6.2.10) for a
comprehensive treatment. In this chapter, we focus on statements and
proofs that do not require selecting maximizers: operator
well-definedness on bounded measurable functions and contraction in
\(\|\cdot\|_\infty\).

\begin{center}\rule{0.5\linewidth}{0.5pt}\end{center}

\subsection{3.5 Bellman Equations}\label{bellman-equations}

The Bellman equations provide \textbf{recursive characterizations} of
value functions. These are the cornerstone of RL theory.

\textbf{Theorem 3.5.1} (Bellman Expectation Equation)
\phantomsection\label{THM-3.5.1-Bellman}

For any policy \(\pi\), the value function \(V^\pi\) satisfies: \[
V^\pi(s)
= \int_{\mathcal{A}}\left[r(s,a) + \gamma \int_{\mathcal{S}} V^\pi(s')\,P(ds' \mid s,a)\right]\pi(da\mid s),
\tag{3.7}
\label{EQ-3.7}\]

where \(r(s,a) = \int_{\mathcal{S}} R(s,a,s')\,P(ds'\mid s,a)\) as in
\hyperref[DEF-3.4.1]{3.4.1}.

Equivalently, in operator notation: \[
V^\pi = \mathcal{T}^\pi V^\pi,
\tag{3.8}
\label{EQ-3.8}\]

where \(\mathcal{T}^\pi\) is the \textbf{Bellman expectation operator}
for policy \(\pi\): \[
(\mathcal{T}^\pi V)(s)
:= \int_{\mathcal{A}}\left[r(s,a) + \gamma \int_{\mathcal{S}} V(s')\,P(ds' \mid s,a)\right]\pi(da\mid s).
\tag{3.9}
\label{EQ-3.9}\]

\emph{Proof.} \textbf{Step 1} (Decompose the return). By definition
\eqref{EQ-3.2}, \[
V^\pi(s) = \mathbb{E}^\pi\left[\sum_{t=0}^\infty \gamma^t R_t \,\bigg|\, S_0 = s\right].
\]

Separate the first reward from the tail: \[
V^\pi(s) = \mathbb{E}^\pi\left[R_0 + \gamma \sum_{t=1}^\infty \gamma^{t-1} R_t \,\bigg|\, S_0 = s\right].
\]

\textbf{Step 2} (Tower property). Apply the law of total expectation
(tower property) conditioning on \((A_0, S_1)\): \[
V^\pi(s) = \mathbb{E}^\pi\left[\mathbb{E}^\pi\left[R_0 + \gamma \sum_{t=1}^\infty \gamma^{t-1} R_t \,\bigg|\, S_0=s, A_0, S_1\right]\right].
\]

\textbf{Step 3} (Markov property). Since \(R_0 = R(S_0, A_0, S_1)\) is
determined by \((S_0, A_0, S_1)\), and future rewards
\((R_1, R_2, \ldots)\) depend only on \(S_1\) onward (Markov property
{[}ASM-3.4.1{]}), we have: \[
\mathbb{E}^\pi\left[R_0 + \gamma \sum_{t=1}^\infty \gamma^{t-1} R_t \,\bigg|\, S_0=s, A_0=a, S_1=s'\right] = R(s,a,s') + \gamma V^\pi(s').
\]

\textbf{Step 4} (Integrate over actions and next states). Taking
expectations over \(A_0 \sim \pi(\cdot\mid s)\) and
\(S_1 \sim P(\cdot\mid s,a)\): \[
V^\pi(s) = \int_{\mathcal{A}}\int_{\mathcal{S}} \left[R(s,a,s') + \gamma V^\pi(s')\right]\,P(ds'\mid s,a)\,\pi(da\mid s).
\]

Rearranging: \[
V^\pi(s) = \int_{\mathcal{A}}\left[\underbrace{\int_{\mathcal{S}} R(s,a,s')\,P(ds'\mid s,a)}_{=: r(s,a)} + \gamma \int_{\mathcal{S}} V^\pi(s')\,P(ds'\mid s,a)\right]\pi(da\mid s),
\] which is \eqref{EQ-3.7}. \(\square\)

\textbf{Remark 3.5.1} (The dynamic programming principle). The proof
uses the \textbf{principle of optimality}: breaking the infinite-horizon
return into immediate reward plus discounted future value. This is the
essence of dynamic programming. The Markov property
\hyperref[ASM-3.4.1]{3.4.1} is crucial---without it, \(V^\pi(s')\) would
depend on the history leading to \(s'\), and the recursion would fail.

\textbf{Theorem 3.5.2} (Bellman Optimality Equation)
\phantomsection\label{THM-3.5.2-Bellman}

The optimal value function \(V^*\) satisfies: \[
V^*(s) = \sup_{a \in \mathcal{A}}\left[r(s,a) + \gamma \int_{\mathcal{S}} V^*(s')\,P(ds'\mid s,a)\right],
\tag{3.10}
\label{EQ-3.10}\]

or in operator notation: \[
V^* = \mathcal{T} V^*,
\tag{3.11}
\label{EQ-3.11}\]

where \(\mathcal{T}\) is the \textbf{Bellman optimality operator}: \[
(\mathcal{T} V)(s) := \sup_{a \in \mathcal{A}}\left[r(s,a) + \gamma \int_{\mathcal{S}} V(s')\,P(ds'\mid s,a)\right].
\tag{3.12}
\label{EQ-3.12}\]

\emph{Proof.} \textbf{Step 1} (Control dominates evaluation). For any
bounded measurable function \(V\) and any policy \(\pi\), we have, for
each \(s\in\mathcal{S}\), \[
(\mathcal{T}^\pi V)(s)
= \int_{\mathcal{A}}\left[r(s,a) + \gamma \int_{\mathcal{S}} V(s')\,P(ds'\mid s,a)\right]\pi(da\mid s)
\le \sup_{a\in\mathcal{A}}\left[r(s,a) + \gamma \int_{\mathcal{S}} V(s')\,P(ds'\mid s,a)\right]
= (\mathcal{T}V)(s).
\]

\textbf{Step 2} (Fixed point characterization of the optimal value).
Section 3.7 shows that \(\mathcal{T}\) is a \(\gamma\)-contraction on
\((B_b(\mathcal{S}),\|\cdot\|_\infty)\), hence has a unique fixed point
\(\bar V\) by \hyperref[THM-3.6.2-Banach]{3.6.2}. Similarly, for each
policy \(\pi\), the evaluation operator \(\mathcal{T}^\pi\) is a
\(\gamma\)-contraction (the proof is the same as for
\hyperref[THM-3.7.1]{3.7.1}, without the supremum), hence has a unique
fixed point \(V^\pi\).

By Step 1, \((\mathcal{T}^\pi)^k V \le \mathcal{T}^k V\) for all \(k\)
and all \(V\in B_b(\mathcal{S})\). Taking \(k\to\infty\) yields
\(V^\pi \le \bar V\), hence \(\sup_\pi V^\pi \le \bar V\) pointwise.

Discounted dynamic-programming theory shows that the fixed point
\(\bar V\) coincides with the optimal value function \(V^*\) defined in
\eqref{EQ-3.5}, and therefore satisfies \(V^*=\mathcal{T}V^*\); see
(Puterman 2014, Theorem 6.2.10) for the measurable-selection details
behind this identification. This gives \eqref{EQ-3.11} and the pointwise
form \eqref{EQ-3.10}. \(\square\)

\textbf{Note on proof structure.} This proof invokes
\hyperref[THM-3.7.1]{3.7.1} (Bellman contraction) and
\hyperref[THM-3.6.2-Banach]{3.6.2} (Banach fixed-point), which we
establish in Section 3.6--3.7. We state the optimality equation here
because it is conceptually fundamental---\emph{this is the equation RL
algorithms solve}. The existence and uniqueness of \(V^*\) follow once
we prove \(\mathcal{T}\) is a contraction in Section 3.7.

\textbf{Remark 3.5.2} (Suprema, maximizers, and greedy policies).
Equation \eqref{EQ-3.10} is stated with \(\sup\) because the supremum
need not be attained without additional assumptions. In finite action
spaces, or under compactness/upper-semicontinuity conditions, the
supremum is attained and we may write \(\max\).

When a measurable maximizer exists, we can extract a deterministic
greedy policy via: \[
\pi^*(s) \in \arg\max_{a \in \mathcal{A}}\left[r(s,a) + \gamma \int_{\mathcal{S}} V^*(s')\,P(ds'\mid s,a)\right].
\] Without measurable maximizers, we work with \(\varepsilon\)-optimal
selectors and interpret \eqref{EQ-3.10} as a value characterization;
Chapter 2 discusses the measurable-selection fine print in more detail.

\textbf{Remark 3.5.3} (CMDPs and regret: where the details live). Many
practical ranking problems impose constraints (e.g., CM2 floors,
exposure parity). \textbf{Constrained MDPs} (CMDPs) handle these by
introducing Lagrange multipliers that convert the constrained problem
into an unconstrained MDP with modified rewards
\(r_\lambda = r - \lambda c\)---the Bellman theory of this section then
applies directly to the relaxed problem. Appendix C develops the full
CMDP framework with rigorous duality and algorithms; see
\hyperref[THM-C.2.1]{C.2.1}, \hyperref[COR-C.3.1]{C.3.1}, and
{[}ALG-C.5.1{]}. Chapter 10 treats constraints operationally as
production guardrails, while Chapter 14 implements soft constraint
optimization via primal--dual methods.

Regret guarantees are developed in Chapter 6 (e.g.,
\hyperref[THM-6.1]{6.1}, {[}THM-6.2{]}) with information-theoretic lower
bounds in Appendix D ({[}THM-D.3.1{]}).

Once we compute \(V^*\) (via value iteration, which we will prove
converges next), extracting the optimal policy is straightforward.

\begin{center}\rule{0.5\linewidth}{0.5pt}\end{center}

\subsection{3.6 Contraction Mappings and the Banach Fixed-Point
Theorem}\label{contraction-mappings-and-the-banach-fixed-point-theorem}

The Bellman operator \(\mathcal{T}\) is a \textbf{contraction mapping}.
This fundamental property guarantees: 1. Existence and uniqueness of the
fixed point \(V^* = \mathcal{T} V^*\) 2. Convergence of value iteration:
\(V_{k+1} = \mathcal{T} V_k \to V^*\) exponentially fast

We now develop this theory rigorously.

\subsubsection{3.6.1 Normed Spaces and
Contractions}\label{normed-spaces-and-contractions}

\textbf{Definition 3.6.1} (Normed Vector Space)
\phantomsection\label{DEF-3.6.1}

A \textbf{normed vector space} is a pair \((V, \|\cdot\|)\) where \(V\)
is a vector space (over \(\mathbb{R}\) or \(\mathbb{C}\)) and
\(\|\cdot\|: V \to \mathbb{R}_+\) is a \textbf{norm} satisfying: 1.
\textbf{Positive definiteness}: \(\|v\| = 0 \iff v = 0\) 2.
\textbf{Homogeneity}: \(\|\alpha v\| = |\alpha| \|v\|\) for all scalars
\(\alpha\) 3. \textbf{Triangle inequality}:
\(\|u + v\| \leq \|u\| + \|v\|\) for all \(u, v \in V\)

\textbf{Definition 3.6.2} (Supremum Norm)
\phantomsection\label{DEF-3.6.2}

For bounded measurable functions \(f: \mathcal{S} \to \mathbb{R}\), the
\textbf{supremum norm} (or \textbf{\(\infty\)-norm}) is: \[
\|f\|_\infty := \sup_{s \in \mathcal{S}} |f(s)|.
\tag{3.13}
\label{EQ-3.13}\]

We write \(B_b(\mathcal{S})\) for the space of bounded measurable
functions: \[
B_b(\mathcal{S}) := \{f: \mathcal{S} \to \mathbb{R} \text{ measurable} : \|f\|_\infty < \infty\}.
\]

Then \((B_b(\mathcal{S}), \|\cdot\|_\infty)\) is a normed vector space.

\textbf{Proposition 3.6.1} (Completeness of
\((B_b(\mathcal{S}), \|\cdot\|_\infty)\))
\phantomsection\label{PROP-3.6.1}

The space \((B_b(\mathcal{S}), \|\cdot\|_\infty)\) is \textbf{complete}:
every Cauchy sequence converges.

\emph{Proof.} \textbf{Step 1} (Cauchy implies pointwise Cauchy). Let
\((f_n)\) be a Cauchy sequence in \(B_b(\mathcal{S})\). For each
\(s \in \mathcal{S}\), \[
|f_n(s) - f_m(s)| \leq \|f_n - f_m\|_\infty \to 0 \quad \text{as } n, m \to \infty.
\] Thus \((f_n(s))\) is a Cauchy sequence in \(\mathbb{R}\). Since
\(\mathbb{R}\) is complete, \(f_n(s) \to f(s)\) for some
\(f(s) \in \mathbb{R}\).

\textbf{Step 2} (Uniform boundedness and measurability). Since \((f_n)\)
is Cauchy, it is bounded: \(\sup_n \|f_n\|_\infty \leq M < \infty\).
Thus \(|f(s)| = \lim_n |f_n(s)| \leq M\) for all \(s\), so
\(\|f\|_\infty \leq M < \infty\). Since each \(f_n\) is measurable and
\(f_n \to f\) pointwise, the limit \(f\) is measurable.

\textbf{Step 3} (Uniform convergence). Given \(\epsilon > 0\), choose
\(N\) such that \(\|f_n - f_m\|_\infty < \epsilon\) for all
\(n, m \geq N\). Fixing \(n \geq N\) and taking \(m \to \infty\): \[
|f_n(s) - f(s)| = \lim_{m \to \infty} |f_n(s) - f_m(s)| \leq \epsilon \quad \text{for all } s.
\] Thus \(\|f_n - f\|_\infty \leq \epsilon\) for all \(n \geq N\),
proving \(f_n \to f\) in \(\|\cdot\|_\infty\). \(\square\)

\textbf{Remark 3.6.1} (Banach spaces and uniform convergence). A
complete normed space is called a \textbf{Banach space}. Proposition
3.6.1 shows \(B_b(\mathcal{S})\) is a Banach space---this is essential
for applying the Banach fixed-point theorem.

A crucial subtlety: Step 3 establishes \textbf{uniform convergence},
where \(\sup_s |f_n(s) - f(s)| \to 0\). This is strictly stronger than
\textbf{pointwise convergence} (where each \(f_n(s) \to f(s)\)
individually, which Step 1 provides). The space of bounded functions is
complete under uniform convergence but \emph{not} under pointwise
convergence---a sequence of bounded continuous functions can converge
pointwise to an unbounded or discontinuous function. This distinction
matters: the Banach fixed-point theorem requires completeness in the
norm topology, and value iteration convergence guarantees
\hyperref[COR-3.7.3]{3.7.3} are statements about uniform convergence
over all states.

\textbf{Definition 3.6.3} (Contraction Mapping)
\phantomsection\label{DEF-3.6.3}

Let \((V, \|\cdot\|)\) be a normed space. A mapping \(T: V \to V\) is a
\textbf{\(\gamma\)-contraction} if there exists \(\gamma \in [0, 1)\)
such that: \[
\|T(f) - T(g)\| \leq \gamma \|f - g\| \quad \text{for all } f, g \in V.
\tag{3.14}
\label{EQ-3.14}\]

\textbf{Intuition}: \(T\) brings points closer together by a factor
\(\gamma < 1\). The distance between \(T(f)\) and \(T(g)\) is strictly
smaller than the distance between \(f\) and \(g\) (unless \(f = g\)).

\textbf{Example 3.6.1} (Scalar contraction). Let \(V = \mathbb{R}\) with
norm \(|x|\). Define \(T(x) = \frac{1}{2}x + 1\). Then: \[
|T(x) - T(y)| = \left|\frac{1}{2}(x - y)\right| = \frac{1}{2}|x - y|,
\] so \(T\) is a \(1/2\)-contraction.

\textbf{Non-Example 3.6.2} (Expansion). Define \(T(x) = 2x\). Then
\(|T(x) - T(y)| = 2|x - y|\), so \(T\) is an \textbf{expansion}, not a
contraction.

\begin{center}\rule{0.5\linewidth}{0.5pt}\end{center}

\subsubsection{3.6.2 Banach Fixed-Point
Theorem}\label{banach-fixed-point-theorem}

\textbf{Theorem 3.6.2} (Banach Fixed-Point Theorem)
\phantomsection\label{THM-3.6.2-Banach}

Let \((V, \|\cdot\|)\) be a complete normed space (Banach space) and
\(T: V \to V\) a \(\gamma\)-contraction with \(\gamma \in [0, 1)\).
Then:

\begin{enumerate}
\def\labelenumi{\arabic{enumi}.}
\tightlist
\item
  \textbf{Existence}: \(T\) has a \textbf{unique fixed point}
  \(v^* \in V\) satisfying \(T(v^*) = v^*\)
\item
  \textbf{Convergence}: For any initial point \(v_0 \in V\), the
  sequence \(v_{k+1} = T(v_k)\) converges to \(v^*\)
\item
  \textbf{Rate}: The convergence is \textbf{exponential}: \[
  \|v_k - v^*\| \leq \frac{\gamma^k}{1 - \gamma} \|T(v_0) - v_0\|
  \tag{3.15}
  \] \{\#EQ-3.15\}
\end{enumerate}

\emph{Proof.} We prove each claim step-by-step.

\textbf{Proof of (1): Uniqueness} Suppose \(T(v^*) = v^*\) and
\(T(w^*) = w^*\) are two fixed points. Then: \[
\|v^* - w^*\| = \|T(v^*) - T(w^*)\| \leq \gamma \|v^* - w^*\|.
\] Since \(\gamma < 1\), this implies \(\|v^* - w^*\| = 0\), hence
\(v^* = w^*\). \(\square\) (Uniqueness)

\textbf{Proof of (2): Convergence to a fixed point} \textbf{Step 1}
(Sequence is Cauchy). Define \(v_k := T^k(v_0)\) (applying \(T\)
iteratively). For \(k \geq 1\): \[
\|v_{k+1} - v_k\| = \|T(v_k) - T(v_{k-1})\| \leq \gamma \|v_k - v_{k-1}\|.
\]

Iterating this inequality: \[
\|v_{k+1} - v_k\| \leq \gamma^k \|v_1 - v_0\|.
\]

For \(n > m\), by the triangle inequality:
\begin{align}
\|v_n - v_m\| &\leq \sum_{k=m}^{n-1} \|v_{k+1} - v_k\| \\
&\leq \sum_{k=m}^{n-1} \gamma^k \|v_1 - v_0\| \\
&= \gamma^m \frac{1 - \gamma^{n-m}}{1 - \gamma} \|v_1 - v_0\| \\
&\leq \frac{\gamma^m}{1 - \gamma} \|v_1 - v_0\|.
\end{align}

Since \(\gamma < 1\), \(\gamma^m \to 0\) as \(m \to \infty\), so
\((v_k)\) is a Cauchy sequence.

\textbf{Step 2} (Completeness implies convergence). Since \(V\) is
complete, there exists \(v^* \in V\) such that \(v_k \to v^*\).

\textbf{Step 3} (Limit is a fixed point). Since \(T\) is a contraction,
it is continuous. Thus: \[
T(v^*) = T\left(\lim_{k \to \infty} v_k\right) = \lim_{k \to \infty} T(v_k) = \lim_{k \to \infty} v_{k+1} = v^*.
\]

So \(v^*\) is a fixed point. By uniqueness (proved above), it is the
\textbf{unique} fixed point. \(\square\) (Existence and Convergence)

\textbf{Proof of (3): Rate} From Step 1 above, taking \(m = 0\) and
letting \(n \to \infty\): \[
\|v^* - v_0\| \leq \sum_{k=0}^\infty \|v_{k+1} - v_k\| \leq \|v_1 - v_0\| \sum_{k=0}^\infty \gamma^k = \frac{\|v_1 - v_0\|}{1 - \gamma}.
\]

For \(k \geq 1\), applying the contraction property: \[
\|v_k - v^*\| = \|T(v_{k-1}) - T(v^*)\| \leq \gamma \|v_{k-1} - v^*\|.
\]

Iterating: \[
\|v_k - v^*\| \leq \gamma^k \|v_0 - v^*\| \leq \frac{\gamma^k}{1 - \gamma} \|v_1 - v_0\|,
\] which is \eqref{EQ-3.15}. \(\square\) (Rate)

\textbf{Remark 3.6.2} (The key mechanisms). This proof deploys several
fundamental techniques:

\begin{enumerate}
\def\labelenumi{\arabic{enumi}.}
\tightlist
\item
  \textbf{Telescoping series}: Write
  \(\|v_n - v_m\| \leq \sum_{k=m}^{n-1} \|v_{k+1} - v_k\|\) to control
  differences
\item
  \textbf{Geometric series}: Bound
  \(\sum_{k=m}^\infty \gamma^k = \gamma^m / (1 - \gamma)\) using
  \(\gamma < 1\)
\item
  \textbf{Completeness}: Cauchy sequences converge---this is
  \textbf{essential} and fails in incomplete spaces (e.g., rationals
  \(\mathbb{Q}\))
\item
  \textbf{Continuity from contraction}: Contractions are uniformly
  continuous, so limits pass through \(T\)
\end{enumerate}

These techniques will reappear in convergence proofs for TD-learning
(Chapters 8, 12) and stochastic approximation (later chapters).

\textbf{Example 3.6.3} (Failure without completeness). Define
\(T: \mathbb{Q} \to \mathbb{Q}\) by
\(T(x) = (x + 2/x)/2\)---Newton-Raphson iteration for finding
\(\sqrt{2}\). Near \(x = 1.5\), this map is a contraction:
\(|T(x) - T(y)| < 0.5 |x - y|\) for \(x, y \in [1, 2] \cap \mathbb{Q}\).
Starting from \(x_0 = 3/2 \in \mathbb{Q}\), the sequence
\(x_{k+1} = T(x_k)\) remains in \(\mathbb{Q}\) and converges\ldots{} but
to \(\sqrt{2} \notin \mathbb{Q}\). The fixed point exists in
\(\mathbb{R}\) but not in the incomplete space \(\mathbb{Q}\). This is
why completeness is essential for {[}THM-3.6.2-Banach{]}---and why we
need \(B_b(\mathcal{S})\) to be a Banach space for value iteration to
converge to a \emph{valid} value function.

\textbf{Remark 3.6.3} (The \(1/(1-\gamma)\) factor). The bound
\eqref{EQ-3.15} shows the convergence rate depends on \(1/(1-\gamma)\).
When \(\gamma \to 1\) (nearly undiscounted), convergence slows
dramatically---this explains why high-\(\gamma\) RL (e.g.,
\(\gamma = 0.99\)) requires many iterations. The factor \(\gamma^k\)
gives \textbf{exponential convergence}: doubling \(k\) squares the
error.

\begin{center}\rule{0.5\linewidth}{0.5pt}\end{center}

\subsection{3.7 Bellman Operator is a
Contraction}\label{bellman-operator-is-a-contraction}

We now prove the central result: the Bellman optimality operator
\(\mathcal{T}\) is a \(\gamma\)-contraction on
\((B_b(\mathcal{S}), \|\cdot\|_\infty)\).

\textbf{Remark 3.7.0} (Self-mapping property). Before proving
contraction, we verify that \(\mathcal{T}\) maps bounded measurable
functions to bounded measurable functions. Under our standing
assumptions---bounded rewards \(|r(s,a)| \leq R_{\max}\) and discount
\(\gamma < 1\) ({[}DEF-3.4.1{]})---if \(\|V\|_\infty < \infty\), then:
\[
\|\mathcal{T}V\|_\infty
= \sup_{s\in\mathcal{S}}\left|\sup_{a\in\mathcal{A}}\left[r(s,a) + \gamma \int_{\mathcal{S}} V(s')\,P(ds'\mid s,a)\right]\right|
\leq R_{\max} + \gamma \|V\|_\infty < \infty.
\] This establishes boundedness. Measurability of
\(s \mapsto (\mathcal{T}V)(s)\) is immediate in the finite-action case
(where the supremum is a maximum over finitely many measurable
functions) and holds under standard topological hypotheses; see
\hyperref[PROP-2.8.2]{2.8.2} and Chapter 2, §2.8.2 (in particular
{[}THM-2.8.3{]}).

\textbf{Theorem 3.7.1} (Bellman Operator Contraction)
\phantomsection\label{THM-3.7.1}

The Bellman optimality operator
\(\mathcal{T}: B_b(\mathcal{S}) \to B_b(\mathcal{S})\) defined by: \[
(\mathcal{T} V)(s) = \sup_{a \in \mathcal{A}}\left[r(s,a) + \gamma \int_{\mathcal{S}} V(s')\,P(ds'\mid s,a)\right]
\] is a \(\gamma\)-contraction with respect to \(\|\cdot\|_\infty\): \[
\|\mathcal{T} V - \mathcal{T} W\|_\infty \leq \gamma \|V - W\|_\infty \quad \text{for all } V, W \in B_b(\mathcal{S}).
\tag{3.16}
\label{EQ-3.16}\]

\emph{Proof.} \textbf{Step 1} (Non-expansiveness of \(\sup\)). For any
real-valued functions \(f,g\) on \(\mathcal{A}\), \[
\left|\sup_{a\in\mathcal{A}} f(a) - \sup_{a\in\mathcal{A}} g(a)\right|
\le \sup_{a\in\mathcal{A}} |f(a)-g(a)|.
\] Indeed,
\(f(a)\le g(a)+|f(a)-g(a)|\le \sup_{a'}g(a')+\sup_{a'}|f(a')-g(a')|\)
for all \(a\), so taking \(\sup_a\) yields
\(\sup_a f(a)\le \sup_a g(a)+\sup_a|f(a)-g(a)|\). Swapping \(f,g\) gives
the reverse inequality, and combining yields the claim.

\textbf{Step 2} (Pointwise contraction). Fix \(s\in\mathcal{S}\) and
define, for each \(a\in\mathcal{A}\), \[
F_V(a) := r(s,a) + \gamma \int_{\mathcal{S}} V(s')\,P(ds'\mid s,a),\qquad
F_W(a) := r(s,a) + \gamma \int_{\mathcal{S}} W(s')\,P(ds'\mid s,a).
\] Then \((\mathcal{T}V)(s)=\sup_{a}F_V(a)\) and
\((\mathcal{T}W)(s)=\sup_{a}F_W(a)\). By Step 1, \[
\left|(\mathcal{T}V)(s) - (\mathcal{T}W)(s)\right|
\le \sup_{a\in\mathcal{A}} |F_V(a)-F_W(a)|
= \gamma \sup_{a\in\mathcal{A}}\left|\int_{\mathcal{S}} (V-W)(s')\,P(ds'\mid s,a)\right|.
\] Since \(P(\cdot\mid s,a)\) is a probability measure and \(V-W\) is
bounded, \[
\left|\int_{\mathcal{S}} (V-W)(s')\,P(ds'\mid s,a)\right|
\le \int_{\mathcal{S}} |V(s')-W(s')|\,P(ds'\mid s,a)
\le \|V-W\|_\infty.
\] Therefore
\(|(\mathcal{T}V)(s)-(\mathcal{T}W)(s)|\le \gamma\|V-W\|_\infty\) for
all \(s\).

\textbf{Step 3} (Supremum over states). Taking
\(\sup_{s\in\mathcal{S}}\) yields
\(\|\mathcal{T}V-\mathcal{T}W\|_\infty\le \gamma\|V-W\|_\infty\), which
is \eqref{EQ-3.16}. \(\square\)

\textbf{Remark 3.7.1} (The sup-stability mechanism). The proof exploits
the \textbf{non-expansiveness of \(\sup\)}: taking a supremum is a
1-Lipschitz operation. Formally, for any functions \(f, g\), \[
|\sup_a f(a) - \sup_a g(a)| \leq \sup_a |f(a) - g(a)|.
\] This is a fundamental technique in dynamic programming theory,
appearing in proofs of policy improvement theorems and error propagation
bounds.

\textbf{Remark 3.7.2} (Norm specificity). The contraction
\eqref{EQ-3.16} holds specifically in the \textbf{sup-norm}
\(\|\cdot\|_\infty\). The Bellman operator is generally \textbf{not} a
contraction in \(L^1\) or \(L^2\) norms---the proof crucially uses \[
\int_{\mathcal{S}} |V(s') - W(s')|\,P(ds'\mid s,a) \leq \|V - W\|_\infty,
\] which fails for other \(L^p\) norms. This norm choice has practical
implications: error bounds in RL propagate through the
\(\|\cdot\|_\infty\) norm, meaning worst-case state errors matter most.

\textbf{Corollary 3.7.2} (Existence and Uniqueness of \(V^*\))
\phantomsection\label{COR-3.7.2}

There exists a unique \(V^* \in B_b(\mathcal{S})\) satisfying the
Bellman optimality equation \(V^* = \mathcal{T} V^*\).

\emph{Proof.} Immediate from Theorems 3.6.2 and 3.7.1: \(\mathcal{T}\)
is a \(\gamma\)-contraction on the Banach space
\((B_b(\mathcal{S}), \|\cdot\|_\infty)\), so it has a unique fixed
point. \(\square\)

\textbf{Corollary 3.7.3} (Value Iteration Convergence)
\phantomsection\label{COR-3.7.3}

For any initial value function \(V_0 \in B_b(\mathcal{S})\), the
sequence: \[
V_{k+1} = \mathcal{T} V_k
\tag{3.17}
\label{EQ-3.17}\]

converges to \(V^*\) with exponential rate: \[
\|V_k - V^*\|_\infty \leq \frac{\gamma^k}{1 - \gamma} \|\mathcal{T} V_0 - V_0\|_\infty.
\tag{3.18}
\label{EQ-3.18}\]

\emph{Proof.} Immediate from Theorems 3.6.2 and 3.7.1. \(\square\)

\textbf{Proposition 3.7.4} (Reward perturbation sensitivity)
\phantomsection\label{PROP-3.7.4}

Fix \((\mathcal{S},\mathcal{A},P,\gamma)\) and let \(r\) and
\(\tilde r = r + \Delta r\) be bounded measurable one-step reward
functions. Let \(V^*_r\) and \(V^*_{\tilde r}\) denote the unique fixed
points of the corresponding Bellman optimality operators on
\(B_b(\mathcal{S})\). Then: \[
\|V^*_{\tilde r} - V^*_r\|_\infty \le \frac{\|\Delta r\|_\infty}{1-\gamma}.
\]

\emph{Proof.} Let \(\mathcal{T}_r\) and \(\mathcal{T}_{\tilde r}\)
denote the two Bellman optimality operators. For any
\(V\in B_b(\mathcal{S})\) and any \(s\in\mathcal{S}\), \[
|(\mathcal{T}_{\tilde r}V)(s) - (\mathcal{T}_r V)(s)|
= \left|\sup_{a\in\mathcal{A}} \left[\tilde r(s,a) + \gamma \int_{\mathcal{S}} V(s')\,P(ds'\mid s,a)\right] - \sup_{a\in\mathcal{A}} \left[r(s,a) + \gamma \int_{\mathcal{S}} V(s')\,P(ds'\mid s,a)\right]\right|
\le \sup_{a\in\mathcal{A}} |\tilde r(s,a) - r(s,a)|
\le \|\Delta r\|_\infty,
\] so
\(\|\mathcal{T}_{\tilde r}V - \mathcal{T}_r V\|_\infty \le \|\Delta r\|_\infty\).

Using the fixed point identities \(V^*_r=\mathcal{T}_r V^*_r\) and
\(V^*_{\tilde r}=\mathcal{T}_{\tilde r} V^*_{\tilde r}\) and the
contraction of \(\mathcal{T}_{\tilde r}\), \[
\|V^*_{\tilde r} - V^*_r\|_\infty
= \|\mathcal{T}_{\tilde r}V^*_{\tilde r} - \mathcal{T}_r V^*_r\|_\infty
\le \|\mathcal{T}_{\tilde r}V^*_{\tilde r} - \mathcal{T}_{\tilde r} V^*_r\|_\infty + \|\mathcal{T}_{\tilde r}V^*_r - \mathcal{T}_r V^*_r\|_\infty
\le \gamma \|V^*_{\tilde r} - V^*_r\|_\infty + \|\Delta r\|_\infty.
\] Rearranging yields
\(\|V^*_{\tilde r} - V^*_r\|_\infty \le \|\Delta r\|_\infty/(1-\gamma)\).
\(\square\)

\textbf{Remark 3.7.5} (Practical implications). Corollary 3.7.3
guarantees that \textbf{value iteration always converges}, regardless of
initialization \(V_0\). The rate \eqref{EQ-3.18} shows that after \(k\)
iterations, the error shrinks by \(\gamma^k\). For \(\gamma = 0.9\), we
have \(\gamma^{10} \approx 0.35\); for \(\gamma = 0.99\), we need
\(k \approx 460\) iterations to reduce error by a factor of 100. This
explains why high-discount RL is computationally expensive.

\textbf{Remark 3.7.6} (OPE preview --- Direct Method). Off-policy
evaluation (Chapter 9) can be performed via a \textbf{model-based Direct
Method}: estimate \((\hat P, \hat r)\) and apply the policy Bellman
operator repeatedly under the model to obtain \[
\widehat{V}^{\pi} := \lim_{k\to\infty} (\mathcal{T}^{\pi}_{\hat P, \hat r})^k V_0,
\tag{3.22}
\label{EQ-3.22}\] for any bounded \(V_0\). The contraction property
(with \(\gamma<1\) and bounded \(\hat r\)) guarantees existence and
uniqueness of \(\widehat{V}^{\pi}\). Chapter 9 develops full off-policy
evaluation (IPS, DR, FQE), comparing the Direct Method previewed here to
importance-weighted estimators.

\textbf{Remark 3.7.7} (The deadly triad --- when contraction fails). The
contraction property \hyperref[THM-3.7.1]{3.7.1} guarantees convergence
for \textbf{exact, tabular} value iteration. However, three ingredients
common in deep RL can break this guarantee:

\begin{enumerate}
\def\labelenumi{\arabic{enumi}.}
\tightlist
\item
  \textbf{Function approximation}: Representing \(V\) or \(Q\) via
  neural networks restricts us to a function class \(\mathcal{F}\). The
  composed operator \(\Pi_{\mathcal{F}} \circ \mathcal{T}\)
  (project-then-Bellman) is generally \textbf{not} a contraction.
\item
  \textbf{Bootstrapping}: TD methods update toward \(r + \gamma V(s')\),
  using the current estimate \(V\). Combined with function
  approximation, this can cause divergence.
\item
  \textbf{Off-policy learning}: Learning about one policy while
  following another introduces distribution mismatch.
\end{enumerate}

The combination---function approximation + bootstrapping +
off-policy---is Sutton's \textbf{deadly triad} ((Sutton and Barto 2018,
sec. 11.3)). Classical counterexamples (e.g., Baird's) demonstrate that
the resulting learning dynamics can diverge even with linear function
approximation. Chapter 7 introduces target networks and experience
replay as partial mitigations. The fundamental tension, however, remains
unresolved in theory---deep RL succeeds empirically despite lacking the
contraction guarantees we have established here. Understanding this gap
between theory and practice is a central theme of Part III.

\begin{center}\rule{0.5\linewidth}{0.5pt}\end{center}

\subsection{\texorpdfstring{3.8 Connection to Contextual Bandits
(\(\gamma = 0\))}{3.8 Connection to Contextual Bandits (\textbackslash gamma = 0)}}\label{connection-to-contextual-bandits-gamma-0}

The \textbf{contextual bandit} from Chapter 1 is the special case
\(\gamma = 0\) (no state transitions, immediate rewards only).

\textbf{Definition 3.8.1} (Bandit Bellman Operator)
\phantomsection\label{DEF-3.8.1}

For a contextual bandit with Q-function
\(Q: \mathcal{X} \times \mathcal{A} \to \mathbb{R}\) (the expected
immediate reward from Chapter 1), the \textbf{bandit Bellman operator}
is: \[
(\mathcal{T}_{\text{bandit}} V)(x) := \sup_{a \in \mathcal{A}} Q(x, a).
\tag{3.19}
\label{EQ-3.19}\]

This is precisely the MDP Bellman operator \eqref{EQ-3.12} specialized
to \(\gamma = 0\), since in the bandit setting the one-step reward is
\(r(x,a)=Q(x,a)\) (and when \(\mathcal{A}\) is finite the supremum is a
maximum): \[
(\mathcal{T} V)(x) = \sup_a [r(x, a) + \gamma \cdot 0] = \sup_a Q(x, a) = (\mathcal{T}_{\text{bandit}} V)(x).
\] In particular, the right-hand side does not depend on \(V\): bandits
have no bootstrapping term, because there is no next-state value to
propagate.

\textbf{Proposition 3.8.1} (Bandit operator fixed point in one
iteration) \phantomsection\label{PROP-3.8.1}

For bandits (\(\gamma = 0\)), the optimal value function is: \[
V^*(x) = \sup_{a \in \mathcal{A}} Q(x, a),
\tag{3.20}
\label{EQ-3.20}\]

and value iteration converges in \textbf{one step}:
\(V_1 = \mathcal{T}_{\text{bandit}} V_0 = V^*\) for any \(V_0\).

\emph{Proof.} Since \(\gamma = 0\), applying the Bellman operator: \[
(\mathcal{T}_{\text{bandit}} V)(x) = \sup_a Q(x, a) = V^*(x),
\] independent of \(V\). Thus \(V_1 = V^*\) for any \(V_0\). \(\square\)

\textbf{Remark 3.8.1} (Contrast with MDPs). For \(\gamma > 0\), value
iteration requires multiple steps because we must propagate value
information backward through state transitions. For bandits, there are
no state transitions---rewards are immediate---so the optimal value is
the statewise supremum of the immediate \(Q\)-values. This is why
Chapter 1 could focus on \textbf{learning \(Q(x, a)\)} without
explicitly constructing value functions.

\textbf{Remark 3.8.2} (Chapter 1 formulation). Recall from Chapter 1 the
bandit optimality condition: the optimal value \eqref{EQ-1.9} is
attained by the greedy policy \eqref{EQ-1.10}, yielding \[
V^*(x) = \max_{a \in \mathcal{A}} Q(x, a), \quad Q(x, a) = \mathbb{E}_\omega[R(x, a, \omega)].
\]

This is exactly \eqref{EQ-3.20}. The bandit formulation is the
\(\gamma = 0\) MDP.

\begin{center}\rule{0.5\linewidth}{0.5pt}\end{center}

\subsection{3.9 Computational
Verification}\label{computational-verification}

We now implement value iteration and verify the convergence theory
numerically.

\subsubsection{3.9.1 Toy MDP: GridWorld
Navigation}\label{toy-mdp-gridworld-navigation}

\textbf{Setup}: A \(5 \times 5\) grid. Agent starts at \((0, 0)\), goal
is \((4, 4)\). Actions:
\(\{\text{up}, \text{down}, \text{left}, \text{right}\}\). Rewards:
\(+10\) at goal, \(-1\) per step (encourages shortest paths).
Transitions: deterministic (move in chosen direction unless blocked by
boundary).

\begin{Shaded}
\begin{Highlighting}[]
\ImportTok{from}\NormalTok{ \_\_future\_\_ }\ImportTok{import}\NormalTok{ annotations}

\ImportTok{from}\NormalTok{ dataclasses }\ImportTok{import}\NormalTok{ dataclass}
\ImportTok{from}\NormalTok{ typing }\ImportTok{import}\NormalTok{ Tuple}

\ImportTok{import}\NormalTok{ numpy }\ImportTok{as}\NormalTok{ np}


\AttributeTok{@dataclass}
\KeywordTok{class}\NormalTok{ GridWorldConfig:}
\NormalTok{    size: }\BuiltInTok{int} \OperatorTok{=} \DecValTok{5}
\NormalTok{    gamma: }\BuiltInTok{float} \OperatorTok{=} \FloatTok{0.9}
\NormalTok{    goal\_reward: }\BuiltInTok{float} \OperatorTok{=} \FloatTok{10.0}


\KeywordTok{class}\NormalTok{ GridWorldMDP:}
    \CommentTok{"""Deterministic GridWorld used in Section 3.9.1."""}

    \KeywordTok{def} \FunctionTok{\_\_init\_\_}\NormalTok{(}\VariableTok{self}\NormalTok{, cfg: GridWorldConfig }\OperatorTok{|} \VariableTok{None} \OperatorTok{=} \VariableTok{None}\NormalTok{) }\OperatorTok{{-}\textgreater{}} \VariableTok{None}\NormalTok{:}
        \VariableTok{self}\NormalTok{.cfg }\OperatorTok{=}\NormalTok{ cfg }\KeywordTok{or}\NormalTok{ GridWorldConfig()}
        \VariableTok{self}\NormalTok{.size }\OperatorTok{=} \VariableTok{self}\NormalTok{.cfg.size}
        \VariableTok{self}\NormalTok{.gamma }\OperatorTok{=} \VariableTok{self}\NormalTok{.cfg.gamma}
        \VariableTok{self}\NormalTok{.goal\_reward }\OperatorTok{=} \VariableTok{self}\NormalTok{.cfg.goal\_reward}

        \VariableTok{self}\NormalTok{.goal }\OperatorTok{=}\NormalTok{ (}\VariableTok{self}\NormalTok{.size }\OperatorTok{{-}} \DecValTok{1}\NormalTok{, }\VariableTok{self}\NormalTok{.size }\OperatorTok{{-}} \DecValTok{1}\NormalTok{)}
        \VariableTok{self}\NormalTok{.n\_states }\OperatorTok{=} \VariableTok{self}\NormalTok{.size }\OperatorTok{*} \VariableTok{self}\NormalTok{.size}
        \VariableTok{self}\NormalTok{.n\_actions }\OperatorTok{=} \DecValTok{4}  \CommentTok{\# up, down, left, right}

        \VariableTok{self}\NormalTok{.P }\OperatorTok{=}\NormalTok{ np.zeros((}\VariableTok{self}\NormalTok{.n\_states, }\VariableTok{self}\NormalTok{.n\_actions, }\VariableTok{self}\NormalTok{.n\_states))}
        \VariableTok{self}\NormalTok{.r }\OperatorTok{=}\NormalTok{ np.zeros((}\VariableTok{self}\NormalTok{.n\_states, }\VariableTok{self}\NormalTok{.n\_actions))}

        \ControlFlowTok{for}\NormalTok{ i }\KeywordTok{in} \BuiltInTok{range}\NormalTok{(}\VariableTok{self}\NormalTok{.size):}
            \ControlFlowTok{for}\NormalTok{ j }\KeywordTok{in} \BuiltInTok{range}\NormalTok{(}\VariableTok{self}\NormalTok{.size):}
\NormalTok{                s }\OperatorTok{=} \VariableTok{self}\NormalTok{.\_state\_index(i, j)}
                \ControlFlowTok{if}\NormalTok{ (i, j) }\OperatorTok{==} \VariableTok{self}\NormalTok{.goal:}
                    \ControlFlowTok{for}\NormalTok{ a }\KeywordTok{in} \BuiltInTok{range}\NormalTok{(}\VariableTok{self}\NormalTok{.n\_actions):}
                        \VariableTok{self}\NormalTok{.P[s, a, s] }\OperatorTok{=} \FloatTok{1.0}
                        \VariableTok{self}\NormalTok{.r[s, a] }\OperatorTok{=} \VariableTok{self}\NormalTok{.goal\_reward}
                    \ControlFlowTok{continue}
                \ControlFlowTok{for}\NormalTok{ a }\KeywordTok{in} \BuiltInTok{range}\NormalTok{(}\VariableTok{self}\NormalTok{.n\_actions):}
\NormalTok{                    i\_next, j\_next }\OperatorTok{=} \VariableTok{self}\NormalTok{.\_next\_state(i, j, a)}
\NormalTok{                    s\_next }\OperatorTok{=} \VariableTok{self}\NormalTok{.\_state\_index(i\_next, j\_next)}
                    \VariableTok{self}\NormalTok{.P[s, a, s\_next] }\OperatorTok{=} \FloatTok{1.0}
                    \VariableTok{self}\NormalTok{.r[s, a] }\OperatorTok{=} \OperatorTok{{-}}\FloatTok{1.0}

    \KeywordTok{def}\NormalTok{ \_state\_index(}\VariableTok{self}\NormalTok{, i: }\BuiltInTok{int}\NormalTok{, j: }\BuiltInTok{int}\NormalTok{) }\OperatorTok{{-}\textgreater{}} \BuiltInTok{int}\NormalTok{:}
        \ControlFlowTok{return}\NormalTok{ i }\OperatorTok{*} \VariableTok{self}\NormalTok{.size }\OperatorTok{+}\NormalTok{ j}

    \KeywordTok{def}\NormalTok{ \_next\_state(}\VariableTok{self}\NormalTok{, i: }\BuiltInTok{int}\NormalTok{, j: }\BuiltInTok{int}\NormalTok{, action: }\BuiltInTok{int}\NormalTok{) }\OperatorTok{{-}\textgreater{}}\NormalTok{ Tuple[}\BuiltInTok{int}\NormalTok{, }\BuiltInTok{int}\NormalTok{]:}
        \ControlFlowTok{if}\NormalTok{ action }\OperatorTok{==} \DecValTok{0}\NormalTok{:  }\CommentTok{\# up}
            \ControlFlowTok{return} \BuiltInTok{max}\NormalTok{(i }\OperatorTok{{-}} \DecValTok{1}\NormalTok{, }\DecValTok{0}\NormalTok{), j}
        \ControlFlowTok{if}\NormalTok{ action }\OperatorTok{==} \DecValTok{1}\NormalTok{:  }\CommentTok{\# down}
            \ControlFlowTok{return} \BuiltInTok{min}\NormalTok{(i }\OperatorTok{+} \DecValTok{1}\NormalTok{, }\VariableTok{self}\NormalTok{.size }\OperatorTok{{-}} \DecValTok{1}\NormalTok{), j}
        \ControlFlowTok{if}\NormalTok{ action }\OperatorTok{==} \DecValTok{2}\NormalTok{:  }\CommentTok{\# left}
            \ControlFlowTok{return}\NormalTok{ i, }\BuiltInTok{max}\NormalTok{(j }\OperatorTok{{-}} \DecValTok{1}\NormalTok{, }\DecValTok{0}\NormalTok{)}
        \ControlFlowTok{return}\NormalTok{ i, }\BuiltInTok{min}\NormalTok{(j }\OperatorTok{+} \DecValTok{1}\NormalTok{, }\VariableTok{self}\NormalTok{.size }\OperatorTok{{-}} \DecValTok{1}\NormalTok{)  }\CommentTok{\# right}

    \KeywordTok{def}\NormalTok{ bellman\_operator(}\VariableTok{self}\NormalTok{, values: np.ndarray) }\OperatorTok{{-}\textgreater{}}\NormalTok{ np.ndarray:}
\NormalTok{        q\_values }\OperatorTok{=} \VariableTok{self}\NormalTok{.r }\OperatorTok{+} \VariableTok{self}\NormalTok{.gamma }\OperatorTok{*}\NormalTok{ np.einsum(}\StringTok{"ijk,k{-}\textgreater{}ij"}\NormalTok{, }\VariableTok{self}\NormalTok{.P, values)}
        \ControlFlowTok{return}\NormalTok{ np.}\BuiltInTok{max}\NormalTok{(q\_values, axis}\OperatorTok{=}\DecValTok{1}\NormalTok{)}

    \KeywordTok{def}\NormalTok{ value\_iteration(}
        \VariableTok{self}\NormalTok{,}
\NormalTok{        V\_init: np.ndarray }\OperatorTok{|} \VariableTok{None} \OperatorTok{=} \VariableTok{None}\NormalTok{,}
        \OperatorTok{*}\NormalTok{,}
\NormalTok{        max\_iter: }\BuiltInTok{int} \OperatorTok{=} \DecValTok{256}\NormalTok{,}
\NormalTok{        tol: }\BuiltInTok{float} \OperatorTok{=} \FloatTok{1e{-}10}\NormalTok{,}
\NormalTok{    ) }\OperatorTok{{-}\textgreater{}}\NormalTok{ Tuple[np.ndarray, }\BuiltInTok{list}\NormalTok{[}\BuiltInTok{float}\NormalTok{]]:}
\NormalTok{        values }\OperatorTok{=}\NormalTok{ np.zeros(}\VariableTok{self}\NormalTok{.n\_states) }\ControlFlowTok{if}\NormalTok{ V\_init }\KeywordTok{is} \VariableTok{None} \ControlFlowTok{else}\NormalTok{ V\_init.copy()}
\NormalTok{        errors: }\BuiltInTok{list}\NormalTok{[}\BuiltInTok{float}\NormalTok{] }\OperatorTok{=}\NormalTok{ []}
        \ControlFlowTok{for}\NormalTok{ \_ }\KeywordTok{in} \BuiltInTok{range}\NormalTok{(max\_iter):}
\NormalTok{            updated }\OperatorTok{=} \VariableTok{self}\NormalTok{.bellman\_operator(values)}
\NormalTok{            error }\OperatorTok{=} \BuiltInTok{float}\NormalTok{(np.}\BuiltInTok{max}\NormalTok{(np.}\BuiltInTok{abs}\NormalTok{(updated }\OperatorTok{{-}}\NormalTok{ values)))}
\NormalTok{            errors.append(error)}
\NormalTok{            values }\OperatorTok{=}\NormalTok{ updated}
            \ControlFlowTok{if}\NormalTok{ error }\OperatorTok{\textless{}}\NormalTok{ tol:}
                \ControlFlowTok{break}
        \ControlFlowTok{return}\NormalTok{ values, errors}


\KeywordTok{def}\NormalTok{ run\_gridworld\_convergence\_check() }\OperatorTok{{-}\textgreater{}} \VariableTok{None}\NormalTok{:}
\NormalTok{    mdp }\OperatorTok{=}\NormalTok{ GridWorldMDP()}
\NormalTok{    V\_star, errors }\OperatorTok{=}\NormalTok{ mdp.value\_iteration()}

\NormalTok{    start\_state }\OperatorTok{=}\NormalTok{ mdp.\_state\_index(}\DecValTok{0}\NormalTok{, }\DecValTok{0}\NormalTok{)}
\NormalTok{    goal\_state }\OperatorTok{=}\NormalTok{ mdp.\_state\_index(}\OperatorTok{*}\NormalTok{mdp.goal)}
\NormalTok{    expected\_goal }\OperatorTok{=}\NormalTok{ mdp.goal\_reward }\OperatorTok{/}\NormalTok{ (}\FloatTok{1.0} \OperatorTok{{-}}\NormalTok{ mdp.gamma)}

    \BuiltInTok{print}\NormalTok{(}\StringTok{"iters"}\NormalTok{, }\BuiltInTok{len}\NormalTok{(errors), }\StringTok{"final\_err"}\NormalTok{, }\SpecialStringTok{f"}\SpecialCharTok{\{}\NormalTok{errors[}\OperatorTok{{-}}\DecValTok{1}\NormalTok{]}\SpecialCharTok{:.3e\}}\SpecialStringTok{"}\NormalTok{)}
    \BuiltInTok{print}\NormalTok{(}\StringTok{"V\_start"}\NormalTok{, }\SpecialStringTok{f"}\SpecialCharTok{\{}\NormalTok{V\_star[start\_state]}\SpecialCharTok{:.6f\}}\SpecialStringTok{"}\NormalTok{)}
    \BuiltInTok{print}\NormalTok{(}\StringTok{"V\_goal"}\NormalTok{, }\SpecialStringTok{f"}\SpecialCharTok{\{}\NormalTok{V\_star[goal\_state]}\SpecialCharTok{:.6f\}}\SpecialStringTok{"}\NormalTok{, }\StringTok{"expected\_goal"}\NormalTok{, }\SpecialStringTok{f"}\SpecialCharTok{\{}\NormalTok{expected\_goal}\SpecialCharTok{:.6f\}}\SpecialStringTok{"}\NormalTok{)}

\NormalTok{    V\_init }\OperatorTok{=}\NormalTok{ np.zeros(mdp.n\_states)}
\NormalTok{    initial\_gap }\OperatorTok{=} \BuiltInTok{float}\NormalTok{(np.}\BuiltInTok{max}\NormalTok{(np.}\BuiltInTok{abs}\NormalTok{(mdp.bellman\_operator(V\_init) }\OperatorTok{{-}}\NormalTok{ V\_init)))}
    \BuiltInTok{print}\NormalTok{(}\StringTok{"initial\_gap"}\NormalTok{, }\SpecialStringTok{f"}\SpecialCharTok{\{}\NormalTok{initial\_gap}\SpecialCharTok{:.6f\}}\SpecialStringTok{"}\NormalTok{)}

    \BuiltInTok{print}\NormalTok{(}\StringTok{"k  ||V\_\{k+1\}{-}V\_k||\_inf  bound\_from\_EQ\_3\_18  bound\_ok"}\NormalTok{)}
    \ControlFlowTok{for}\NormalTok{ k, err }\KeywordTok{in} \BuiltInTok{enumerate}\NormalTok{(errors[:}\DecValTok{10}\NormalTok{]):}
\NormalTok{        bound }\OperatorTok{=}\NormalTok{ (mdp.gamma}\OperatorTok{**}\NormalTok{k }\OperatorTok{/}\NormalTok{ (}\FloatTok{1.0} \OperatorTok{{-}}\NormalTok{ mdp.gamma)) }\OperatorTok{*}\NormalTok{ initial\_gap}
\NormalTok{        bound\_ok }\OperatorTok{=}\NormalTok{ err }\OperatorTok{\textless{}=}\NormalTok{ bound }\OperatorTok{+} \FloatTok{1e{-}9}
        \BuiltInTok{print}\NormalTok{(}\SpecialStringTok{f"}\SpecialCharTok{\{}\NormalTok{k}\SpecialCharTok{:2d\}}\SpecialStringTok{  }\SpecialCharTok{\{}\NormalTok{err}\SpecialCharTok{:18.10f\}}\SpecialStringTok{  }\SpecialCharTok{\{}\NormalTok{bound}\SpecialCharTok{:16.10f\}}\SpecialStringTok{  }\SpecialCharTok{\{}\NormalTok{bound\_ok}\SpecialCharTok{\}}\SpecialStringTok{"}\NormalTok{)}

\NormalTok{    ratios }\OperatorTok{=}\NormalTok{ [}
\NormalTok{        errors[k] }\OperatorTok{/}\NormalTok{ errors[k }\OperatorTok{{-}} \DecValTok{1}\NormalTok{]}
        \ControlFlowTok{for}\NormalTok{ k }\KeywordTok{in} \BuiltInTok{range}\NormalTok{(}\DecValTok{1}\NormalTok{, }\BuiltInTok{min}\NormalTok{(}\BuiltInTok{len}\NormalTok{(errors), }\DecValTok{25}\NormalTok{))}
        \ControlFlowTok{if}\NormalTok{ errors[k }\OperatorTok{{-}} \DecValTok{1}\NormalTok{] }\OperatorTok{\textgreater{}} \FloatTok{1e{-}12}
\NormalTok{    ]}
\NormalTok{    tail }\OperatorTok{=}\NormalTok{ ratios[}\OperatorTok{{-}}\DecValTok{8}\NormalTok{:]}
    \BuiltInTok{print}\NormalTok{(}\StringTok{"tail\_ratios"}\NormalTok{, }\StringTok{" "}\NormalTok{.join(}\SpecialStringTok{f"}\SpecialCharTok{\{}\NormalTok{r}\SpecialCharTok{:.4f\}}\SpecialStringTok{"} \ControlFlowTok{for}\NormalTok{ r }\KeywordTok{in}\NormalTok{ tail))}

\NormalTok{    grid }\OperatorTok{=}\NormalTok{ V\_star.reshape((mdp.size, mdp.size))}
    \BuiltInTok{print}\NormalTok{(}\StringTok{"grid"}\NormalTok{)}
    \ControlFlowTok{for}\NormalTok{ i }\KeywordTok{in} \BuiltInTok{range}\NormalTok{(mdp.size):}
        \BuiltInTok{print}\NormalTok{(}\StringTok{" "}\NormalTok{.join(}\SpecialStringTok{f"}\SpecialCharTok{\{}\NormalTok{grid[i, j]}\SpecialCharTok{:7.2f\}}\SpecialStringTok{"} \ControlFlowTok{for}\NormalTok{ j }\KeywordTok{in} \BuiltInTok{range}\NormalTok{(mdp.size)))}


\NormalTok{run\_gridworld\_convergence\_check()}
\end{Highlighting}
\end{Shaded}

Output:

\begin{verbatim}
iters 242 final_err 9.386e-11
V_start 37.351393
V_goal 100.000000 expected_goal 100.000000
initial_gap 10.000000
k  ||V_{k+1}-V_k||_inf  bound_from_EQ_3_18  bound_ok
 0       10.0000000000    100.0000000000  True
 1        9.0000000000     90.0000000000  True
 2        8.1000000000     81.0000000000  True
 3        7.2900000000     72.9000000000  True
 4        6.5610000000     65.6100000000  True
 5        5.9049000000     59.0490000000  True
 6        5.3144100000     53.1441000000  True
 7        4.7829690000     47.8296900000  True
 8        4.3046721000     43.0467210000  True
 9        3.8742048900     38.7420489000  True
tail_ratios 0.9000 0.9000 0.9000 0.9000 0.9000 0.9000 0.9000 0.9000
grid
  37.35   42.61   48.46   54.95   62.17
  42.61   48.46   54.95   62.17   70.19
  48.46   54.95   62.17   70.19   79.10
  54.95   62.17   70.19   79.10   89.00
  62.17   70.19   79.10   89.00  100.00
\end{verbatim}

\begin{NoteBox}{Code ↔ Lab (Contraction Verification)}

We verify \hyperref[COR-3.7.3]{3.7.3} and the rate bound \eqref{EQ-3.18}
using the value-iteration listing above. The repository also includes a
regression test that mirrors this computation:
\texttt{tests/ch03/test\_value\_iteration.py}. - Run:
\texttt{.venv/bin/pytest\ -q\ tests/ch03/test\_value\_iteration.py}

\end{NoteBox}

\subsubsection{3.9.2 Analysis}\label{analysis}

The numerical experiment confirms:

\begin{enumerate}
\def\labelenumi{\arabic{enumi}.}
\tightlist
\item
  \textbf{Convergence}: value iteration converges within the configured
  iteration budget, with final update size below the tolerance.
\item
  \textbf{Rate bound check}: the printed quantity
  \(\|V_{k+1}-V_k\|_\infty\) remains below the right-hand side of
  \eqref{EQ-3.18} in the displayed iterations, providing a numerical
  sanity check on the contraction-based rate.
\item
  \textbf{Exponential decay}: consecutive error ratios are essentially
  constant at \(\gamma = 0.9\), matching the contraction mechanism.
\item
  \textbf{Goal-state semantics}: since the goal is absorbing with
  per-step reward \texttt{goal\_reward}, we obtain
  \(V^*(\text{goal})=\text{goal\_reward}/(1-\gamma)\).
\end{enumerate}

\textbf{Key observations}:

\begin{itemize}
\tightlist
\item
  The theoretical bound is \textbf{tight}: observed errors track
  \(\gamma^k\) behavior closely
\item
  Higher \(\gamma\) (closer to 1) implies slower convergence: for
  \(\gamma = 0.99\), convergence requires on the order of hundreds of
  iterations in this GridWorld.
\item
  Value iteration is \textbf{robust}: it converges for any
  initialization \(V_0\) (here \(V_0 \equiv 0\))
\end{itemize}

\begin{center}\rule{0.5\linewidth}{0.5pt}\end{center}

\subsection{3.10 RL Bridges: Previewing Multi-Episode
Dynamics}\label{rl-bridges-previewing-multi-episode-dynamics}

In Chapter 11 we extend the within-session MDP of this chapter to an
inter-session (multi-episode) MDP. Chapter 1's contextual bandit
formalism and the discounted MDP formalism of this chapter both treat a
single session in isolation. In practice, many objectives are
inter-session: actions taken today influence the probability of future
sessions and the distribution of future states.

The multi-episode formulation introduces three concrete changes:

\begin{enumerate}
\def\labelenumi{\arabic{enumi}.}
\tightlist
\item
  Inter-session state transitions: the state includes variables such as
  satisfaction, recency, and loyalty tier, and these evolve across
  sessions as functions of engagement signals (clicks, purchases) and
  exogenous factors (seasonality).
\item
  Retention (hazard) modeling: a probabilistic mechanism decides whether
  another session occurs, based on the current inter-session state.
\item
  Long-term value across sessions: the return sums rewards over
  sessions, not only within a single session.
\end{enumerate}

The operator-theoretic content does not change: once inter-session
dynamics are part of the transition kernel, Bellman operators remain
contractions under discounting, and value iteration remains a
fixed-point method. Conceptually, this clarifies reward design. Chapter
1's reward \eqref{EQ-1.2} includes \(\delta \cdot \text{CLICKS}\) as a
proxy for long-run value; in Chapter 11 we encode engagement into the
state dynamics through retention, so long-run effects are represented
without relying on a separate proxy term.

\begin{NoteBox}{Code ↔ Reward (MOD-zoosim.dynamics.reward)}

Chapter 1's single-step reward \eqref{EQ-1.2} maps to configuration and
aggregation code: - Weights and defaults:
\texttt{zoosim/core/config.py:195} (\texttt{RewardConfig}) - Engagement
weight guardrail (\(\\delta/\\alpha\) bound):
\texttt{zoosim/dynamics/reward.py:56} These safeguards keep \(\delta\)
small and bounded in the MVP regime while we develop multi-episode value
in Chapter 11.

\end{NoteBox}

\begin{NoteBox}{Code ↔ Simulator (MOD-zoosim.multi\_episode.session\_env, MOD-zoosim.multi\_episode.retention)}

Multi-episode transitions and retention are implemented in the
simulator: - Inter-session MDP wrapper:
\texttt{zoosim/multi\_episode/session\_env.py:79}
(\texttt{MultiSessionEnv.step}) - Retention probability (logistic
hazard): \texttt{zoosim/multi\_episode/retention.py:22}
(\texttt{return\_probability}) - Retention config:
\texttt{zoosim/core/config.py:208} (\texttt{RetentionConfig}),
\texttt{zoosim/core/config.py:216} (\texttt{base\_rate}),
\texttt{zoosim/core/config.py:217} (\texttt{click\_weight}),
\texttt{zoosim/core/config.py:218} (\texttt{satisfaction\_weight}) In
this regime, engagement enters via state transitions, aligning with the
long-run objective previewed by \eqref{EQ-1.2-prime}e and Chapter 11.

\end{NoteBox}

\begin{center}\rule{0.5\linewidth}{0.5pt}\end{center}

\subsection{3.11 Summary: What We Have
Built}\label{summary-what-we-have-built}

This chapter established the operator-theoretic foundations of
reinforcement learning:

Stochastic processes (Section 3.2--3.3): - Filtrations
\((\mathcal{F}_t)\) model information accumulation over time - Stopping
times \(\tau\) capture random termination (session abandonment, purchase
events) - Adapted processes ensure causality (policies depend on
history, not future)

Markov Decision Processes (Section 3.4): - Formal tuple
\((\mathcal{S}, \mathcal{A}, P, R, \gamma)\) with standard Borel
assumptions - Value functions \(V^\pi(s)\), \(Q^\pi(s, a)\) as expected
cumulative rewards - Bellman equations \eqref{EQ-3.7} and
\eqref{EQ-3.10} as recursive characterizations

Contraction theory (Section 3.6--3.7): - Banach fixed-point theorem
\hyperref[THM-3.6.2-Banach]{3.6.2} guarantees existence, uniqueness, and
exponential convergence - Bellman operator \(\mathcal{T}\) is a
\(\gamma\)-contraction in sup-norm \hyperref[THM-3.7.1]{3.7.1} - Value
iteration \(V_{k+1} = \mathcal{T} V_k\) converges at rate \(\gamma^k\)
\hyperref[COR-3.7.3]{3.7.3} - Caveat: Contraction fails with function
approximation (deadly triad, Remark 3.7.7)

Connection to bandits (Section 3.8): - Contextual bandits are the
\(\gamma = 0\) special case (no state transitions) - Chapter 1's
formulation ({[}EQ-1.8{]}, \eqref{EQ-1.9}, and {[}EQ-1.10{]}) is
recovered exactly

Numerical verification (Section 3.9): - GridWorld experiment confirms
theoretical convergence rate \eqref{EQ-3.18} - Exponential decay
\(\gamma^k\) observed empirically

What comes next:

\begin{itemize}
\tightlist
\item
  \textbf{Chapter 4--5}: Build the simulator (\texttt{zoosim}) with
  catalog, users, queries, click models
\item
  \textbf{Chapter 6}: Implement LinUCB and Thompson Sampling for
  discrete template bandits
\item
  \textbf{Chapter 7}: Continuous action optimization via \(Q(x, a)\)
  regression
\item
  \textbf{Chapter 9}: Off-policy evaluation (OPE) using importance
  sampling
\item
  \textbf{Chapter 10}: Production guardrails (CM2 floors,
  \(\\Delta\\text{Rank}@k\) stability) applying CMDP theory from Section
  3.5
\item
  \textbf{Chapter 11}: Multi-episode MDPs with retention dynamics
\end{itemize}

All later algorithms---TD-learning, Q-learning, policy gradients---use
Bellman operators as their organizing object, but their convergence
guarantees require additional assumptions and are established
case-by-case in later chapters. In Chapter 3, the contraction property
yields a complete convergence story for exact dynamic programming, and
the fixed-point theorem tells us what value iteration converges to.

\begin{center}\rule{0.5\linewidth}{0.5pt}\end{center}

\subsection{3.12 Exercises}\label{exercises}

\textbf{Exercise 3.1} (Stopping times) {[}15 min{]}

Let \((S_t)\) be a user satisfaction process with \(S_t \in [0, 1]\).
Which of the following are stopping times?

\begin{enumerate}
\def\labelenumi{(\alph{enumi})}
\tightlist
\item
  \(\tau_1 = \inf\{t : S_t < 0.3\}\) (first time satisfaction drops
  below 0.3)
\item
  \(\tau_2 = \sup\{t \leq T : S_t \geq 0.8\}\) (last time satisfaction
  exceeds 0.8 before horizon \(T\))
\item
  \(\tau_3 = \min\{t : S_{t+1} < S_t\}\) (first time satisfaction
  decreases)
\end{enumerate}

Justify the answers using \hyperref[DEF-3.2.4]{3.2.4}.

\textbf{Exercise 3.2} (Bellman equation verification) {[}15 min{]}

Consider a 2-state MDP with \(\mathcal{S} = \{s_1, s_2\}\),
\(\mathcal{A} = \{a_1, a_2\}\), \(\gamma = 0.9\). Transitions and
rewards:
\begin{align}
P(\cdot | s_1, a_1) &= (0.8, 0.2), \quad r(s_1, a_1) = 5 \\
P(\cdot | s_1, a_2) &= (0.2, 0.8), \quad r(s_1, a_2) = 10 \\
P(\cdot | s_2, a_1) &= (0.5, 0.5), \quad r(s_2, a_1) = 2 \\
P(\cdot | s_2, a_2) &= (0.3, 0.7), \quad r(s_2, a_2) = 8
\end{align}

Given \(V(s_1) = 50\), \(V(s_2) = 60\), compute \((\mathcal{T} V)(s_1)\)
and \((\mathcal{T} V)(s_2)\) using \eqref{EQ-3.12}.

\textbf{Exercise 3.3} (Contraction property) {[}20 min{]}

Prove that the Bellman expectation operator \(\mathcal{T}^\pi\) for a
fixed policy \(\pi\) (defined in {[}EQ-3.9{]}) is a
\(\gamma\)-contraction, using a similar argument to
\hyperref[THM-3.7.1]{3.7.1}.

\textbf{Exercise 3.4} (Value iteration implementation) {[}extended: 30
min{]}

Implement value iteration for the GridWorld MDP from Section 3.9.1, but
with \textbf{stochastic transitions}: with probability 0.8, the agent
moves in the intended direction; with probability 0.2, it moves in a
random perpendicular direction. Verify that:

\begin{enumerate}
\def\labelenumi{(\alph{enumi})}
\tightlist
\item
  Value iteration still converges
\item
  The convergence rate satisfies \eqref{EQ-3.18}
\item
  The optimal policy changes (compare to deterministic case)
\end{enumerate}

\subsubsection{Labs}\label{labs}

\begin{itemize}
\tightlist
\item
  \href{./exercises_labs.md\#lab-31--contraction-ratio-tracker}{Lab 3.1
  --- Contraction Ratio Tracker}: execute the GridWorld contraction
  experiment and compare empirical ratios against the \(\gamma\) bound
  in \eqref{EQ-3.16}.
\item
  \href{./exercises_labs.md\#lab-32--value-iteration-wall-clock-profiling}{Lab
  3.2 --- Value Iteration Wall-Clock Profiling}: sweep multiple
  discounts, log iteration counts, and tie the scaling back to
  \hyperref[COR-3.7.3]{3.7.3} (value iteration convergence rate).
\end{itemize}

\textbf{Exercise 3.5} (Bandit special case) {[}10 min{]}

Verify that for \(\gamma = 0\), the Bellman operator \eqref{EQ-3.12}
reduces to the bandit operator \eqref{EQ-3.19}. Explain why value
iteration converges in one step for bandits.

\textbf{Exercise 3.6} (Discount factor exploration) {[}20 min{]}

Using the GridWorld code from Section 3.9.1, run value iteration for
\(\gamma \in \{0.5, 0.7, 0.9, 0.99\}\). Plot the number of iterations
required for convergence (tolerance \(10^{-6}\)) as a function of
\(\gamma\). Explain the relationship using \eqref{EQ-3.18}.

\textbf{Exercise 3.7} (RL preview: Policy evaluation) {[}extended: 30
min{]}

Implement \textbf{policy evaluation} (iterative computation of \(V^\pi\)
for a fixed policy \(\pi\) using {[}EQ-3.8{]}). For the GridWorld MDP:

\begin{enumerate}
\def\labelenumi{(\alph{enumi})}
\tightlist
\item
  Define a suboptimal policy \(\pi\): always go right unless at right
  edge (then go down)
\item
  Compute \(V^\pi\) via policy evaluation:
  \(V_{k+1} = \mathcal{T}^\pi V_k\)
\item
  Compare \(V^\pi\) to \(V^*\) (from value iteration)
\item
  Verify that \(V^\pi(s) \leq V^*(s)\) for all \(s\) (why must this
  hold?)
\end{enumerate}

\begin{center}\rule{0.5\linewidth}{0.5pt}\end{center}

\subsection{References}\label{references}

See \texttt{docs/references.bib} for full citations.

Key references for this chapter: - (Puterman 2014) --- Definitive MDP
textbook (Puterman) - (Bertsekas 2012) --- Dynamic programming and
optimal control (Bertsekas) - (Folland 1999) --- Measure theory and
functional analysis foundations - (Brezis 2011) --- Banach space theory
and operator methods - (Sutton and Barto 2018) --- Modern RL textbook
and the deadly triad discussion

\begin{center}\rule{0.5\linewidth}{0.5pt}\end{center}

\subsection{3.13 Production Checklist}\label{production-checklist}

\begin{TipBox}{Production Checklist (Chapter 3)}

- \textbf{Seeds}: Ensure RNGs for stochastic MDPs use fixed seeds from
\texttt{SimulatorConfig.seed} for reproducibility - \textbf{Discount
factor}: Document \(\gamma\) choice in config files; highlight
\(\gamma \to 1\) convergence slowdown - \textbf{Numerical stability}:
Use double precision (\texttt{float64}) for value iteration to avoid
accumulation errors - \textbf{Cross-references}: Update Knowledge Graph
(\texttt{docs/knowledge\_graph/graph.yaml}) with all theorem/definition
IDs - \textbf{Tests}: Add regression tests for value iteration
convergence (verify \eqref{EQ-3.18} bounds programmatically)

\end{TipBox}

\begin{center}\rule{0.5\linewidth}{0.5pt}\end{center}

\subsection{Exercises \& Labs}\label{exercises-labs}

Companion material for Chapter 3 lives in:

\begin{itemize}
\tightlist
\item
  Exercises and runnable lab prompts:
  \texttt{docs/book/ch03/exercises\_labs.md}
\item
  Worked solutions with printed outputs:
  \texttt{docs/book/ch03/ch03\_lab\_solutions.md}
\end{itemize}

Reproducibility checks:

\begin{itemize}
\tightlist
\item
  Chapter 3 regression test:
  \texttt{.venv/bin/pytest\ -q\ tests/ch03/test\_value\_iteration.py}
\item
  Run all Chapter 3 labs:
  \texttt{.venv/bin/python\ scripts/ch03/lab\_solutions.py\ -\/-all}
\end{itemize}

\protect\phantomsection\label{refs}
\begin{CSLReferences}{1}{1}
\bibitem[\citeproctext]{ref-bertsekas:dp:2012}
Bertsekas, Dimitri P. 2012. \emph{Dynamic Programming and Optimal
Control}. 4th ed. Vol. 1. Athena Scientific.

\bibitem[\citeproctext]{ref-brezis:functional_analysis:2011}
Brezis, Haı̈m. 2011. \emph{Functional Analysis, Sobolev Spaces and
Partial Differential Equations}. Universitext. Springer.

\bibitem[\citeproctext]{ref-folland:real_analysis:1999}
Folland, Gerald B. 1999. \emph{Real Analysis: Modern Techniques and
Their Applications}. 2nd ed. Wiley.

\bibitem[\citeproctext]{ref-puterman:mdps:2014}
Puterman, Martin L. 2014. \emph{Markov Decision Processes: Discrete
Stochastic Dynamic Programming}. Wiley Series in Probability and
Statistics. John Wiley \& Sons.

\bibitem[\citeproctext]{ref-sutton:barto:2018}
Sutton, Richard S., and Andrew G. Barto. 2018. \emph{Reinforcement
Learning: An Introduction}. 2nd ed. MIT Press.

\end{CSLReferences}

\end{document}
